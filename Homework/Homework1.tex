%%%%%%%%%%%%%%%%%%%%%%%%%%%%%%%%%%%%%%%%%%%%%%%%%%%%%%%%%%%%%%%%%%%%%%%%%%%%%%%%%%%%%%%%%%%%%%%%%%%%%%%%%%%%%%%%%%%%%%%%%%%%%%%%%%%%%%%%%%%%%%%%%%%%%%%%%%%%%%%%%%%
% Written By Michael Brodskiy
% Class: Power Systems Analysis
% Professor: M. Onabajo
%%%%%%%%%%%%%%%%%%%%%%%%%%%%%%%%%%%%%%%%%%%%%%%%%%%%%%%%%%%%%%%%%%%%%%%%%%%%%%%%%%%%%%%%%%%%%%%%%%%%%%%%%%%%%%%%%%%%%%%%%%%%%%%%%%%%%%%%%%%%%%%%%%%%%%%%%%%%%%%%%%%

\documentclass[12pt]{article} 
\usepackage{alphalph}
\usepackage[utf8]{inputenc}
\usepackage[russian,english]{babel}
\usepackage{titling}
\usepackage{amsmath}
\usepackage{graphicx}
\usepackage{enumitem}
\usepackage{amssymb}
\usepackage[super]{nth}
\usepackage{everysel}
\usepackage{ragged2e}
\usepackage{geometry}
\usepackage{multicol}
\usepackage{fancyhdr}
\usepackage{cancel}
\usepackage{siunitx}
\usepackage{physics}
\usepackage{tikz}
\usepackage{mathdots}
\usepackage{yhmath}
\usepackage{cancel}
\usepackage{color}
\usepackage{array}
\usepackage{multirow}
\usepackage{gensymb}
\usepackage{tabularx}
\usepackage{extarrows}
\usepackage{booktabs}
\usepackage{lastpage}
\usepackage{float}
\usepackage{listings}
\usetikzlibrary{fadings}
\usetikzlibrary{patterns}
\usetikzlibrary{shadows.blur}
\usetikzlibrary{shapes}

\geometry{top=1.0in,bottom=1.0in,left=1.0in,right=1.0in}
\newcommand{\subtitle}[1]{%
  \posttitle{%
    \par\end{center}
    \begin{center}\large#1\end{center}
    \vskip0.5em}%

}
\usepackage{hyperref}
\hypersetup{
colorlinks=true,
linkcolor=blue,
filecolor=magenta,      
urlcolor=blue,
citecolor=blue,
}


\title{Homework 1}
\date{\today}
\author{Michael Brodskiy\\ \small Professor: A. Abur}

\begin{document}

\maketitle

\begin{enumerate}

    \begin{center}
      \underline{Chapter 2}:
    \end{center}

    \setcounter{enumi}{3}

  \item

    First, we find the angle difference, $\phi$:

    $$\phi_1=\cos^{-1}\left( .707 \right)$$
    $$\phi_1\approx 45^{\circ}$$

    From here, we find the initial load current, based on the provided load voltage and power draw:

    $$V_L=440\left[ \si{\volt} \right]$$
    $$\hat{I}_L=\frac{200\cdot10^3}{\sqrt{3}(44)(.707)}\angle-\phi_1$$
    $$\hat{I}_L=371.19\angle -45^{\circ}$$

    We know that power may be expressed as:

    $$S=P+jQ$$

    We find the reactive power of the load:

    $$Q=P\frac{\sin(\phi_1)}{\cos(\phi_1)}=P\tan(\phi_1)$$
    $$Q=200[\si{\kilo}\text{VAr}]$$

    Adding the capacitors in parallel allows us to sum the net positive VArs:

    $$Q_{net}=200-50=150[\si{\kilo}\text{VAr}]$$

    The new power factor becomes:

    $$\phi_2=\tan^{-1}\left(\frac{150[\si{\kilo}\text{VAr}]}{200[\si{\kilo\watt}]})=\tan^{-1}(.75)=36.87^{\circ}$$
    $$\boxed{pf_2=\cos(36.87)=.8\text{ lagging}}$$

    The current then becomes:

    $$I_{L2}=\frac{200\cdot10^3}{\sqrt{3}(440)(.8)}=328.04$$
    $$\boxed{\hat{I}_{L2}=328.04\angle -36.87^{\circ}}$$

  \item

    \begin{enumerate}

      \item 

        We begin by finding the reactive component of the load:

        $$Q=P\tan(\cos^{-1}(pf))=10\tan(\cos^{-1}(.9))$$
        $$Q=4.843[\si{\kilo}\text{VAr}]$$

        Putting the components together, we get:

        $$S=P+jQ$$
        $$\boxed{S=10+4.843j[\si{\volt\ampere}]}$$

      \item To find the magnitude of the current, we may use the following relation:

        $$P=IV(pf)$$

        We insert known values:

        $$10^4=(I)(416)(.9)$$

        This gives us:

        $$|I|=\frac{10^4}{416(.9)}$$
        $$\boxed{|I|=26.709[\si{\ampere}]}$$

      \item Assuming $\angle I=0$, we can say:

        $$\hat{I}=26.709\angle 0^{\circ}[\si{\ampere}]$$

        And from this we say:

        $$\hat{V}=416\angle\cos^{-1}(.9)$$
        $$\hat{V}=416\angle 25.84^{\circ}[\si{\volt}]$$

        We multiply the two together to find the power:

        $$p(t)=\hat{I}\hat{V}=26.709\angle 0^{\circ}\cdot416\angle 25.84^{\circ}$$
        $$p(t)=11.111\angle25.84^{\circ}[\si{\kilo\watt}]$$

    \end{enumerate}

    \setcounter{enumi}{10}

  \item

    The first step has us converting $\Delta$ connections to `Y' connections:

    $$Z_Y=-\frac{Z_{\Delta}}{3}=-\frac{1}{3}j[\si{\ohm}$$
    $$\hat{V}_{a''n}=\frac{\sqrt{3}}{3}\hat{V}_{ab''}\angle-30^{\circ}=\frac{\sqrt{3}}{3}\angle-30^{\circ}[\si{\volt}]$$

    We may then find values for all of the voltages (only the angles change):

    $$\left[ \begin{matrix} V_{a''n}\\V_{b''n}\\V_{c''n} \end{matrix}\right]=\frac{\sqrt{3}}{3}\angle\left[ \begin{matrix} -30\\-150\\90\end{matrix} \right]^{\circ}[\si{\volt}]$$

      Using equivalence, we can develop a single-phase circuit as follows:

      \begin{figure}[H]
        \centering
        \tikzset{every picture/.style={line width=0.75pt}} %set default line width to 0.75pt        

\begin{tikzpicture}[x=0.75pt,y=0.75pt,yscale=-1,xscale=1]
%uncomment if require: \path (0,437); %set diagram left start at 0, and has height of 437

%Shape: Circle [id:dp25415500658791657] 
\draw   (100,149) .. controls (100,135.19) and (111.19,124) .. (125,124) .. controls (138.81,124) and (150,135.19) .. (150,149) .. controls (150,162.81) and (138.81,174) .. (125,174) .. controls (111.19,174) and (100,162.81) .. (100,149) -- cycle ;
%Straight Lines [id:da8534411808673741] 
\draw    (125,78.29) -- (125,124) ;
%Straight Lines [id:da9051744751164938] 
\draw    (125,174) -- (125,219.71) ;
%Straight Lines [id:da03528415731907275] 
\draw    (170.71,78.29) -- (125,78.29) ;
%Shape: Inductor (Air Core) [id:dp5589699497346402] 
\draw   (170.71,78.29) -- (196.17,78.29) .. controls (196.53,67.78) and (200.07,58.8) .. (205.07,55.66) .. controls (210.07,52.52) and (215.52,55.86) .. (218.79,64.07) .. controls (221.32,70.49) and (222.35,78.77) .. (221.62,86.82) .. controls (221.62,89.96) and (220.35,92.5) .. (218.79,92.5) .. controls (217.23,92.5) and (215.97,89.96) .. (215.97,86.82) .. controls (215.24,78.77) and (216.27,70.49) .. (218.79,64.07) .. controls (221.73,57.25) and (225.83,53.37) .. (230.11,53.37) .. controls (234.39,53.37) and (238.48,57.25) .. (241.42,64.07) .. controls (243.95,70.49) and (244.98,78.77) .. (244.25,86.82) .. controls (244.25,89.96) and (242.98,92.5) .. (241.42,92.5) .. controls (239.86,92.5) and (238.59,89.96) .. (238.59,86.82) .. controls (237.86,78.77) and (238.9,70.49) .. (241.42,64.07) .. controls (244.36,57.25) and (248.45,53.37) .. (252.74,53.37) .. controls (257.02,53.37) and (261.11,57.25) .. (264.05,64.07) .. controls (266.57,70.49) and (267.61,78.77) .. (266.88,86.82) .. controls (266.88,89.96) and (265.61,92.5) .. (264.05,92.5) .. controls (262.49,92.5) and (261.22,89.96) .. (261.22,86.82) .. controls (260.49,78.77) and (261.52,70.49) .. (264.05,64.07) .. controls (267.33,55.86) and (272.77,52.52) .. (277.78,55.66) .. controls (282.78,58.8) and (286.31,67.78) .. (286.68,78.29) -- (312.13,78.29) ;
%Straight Lines [id:da5290413238311188] 
\draw    (312.13,219.71) -- (125,219.71) ;
%Shape: Inductor (Air Core) [id:dp8132326460237946] 
\draw   (312.13,78.29) -- (337.59,78.29) .. controls (337.95,67.78) and (341.49,58.8) .. (346.49,55.66) .. controls (351.49,52.52) and (356.94,55.86) .. (360.22,64.07) .. controls (362.74,70.49) and (363.77,78.77) .. (363.04,86.82) .. controls (363.04,89.96) and (361.78,92.5) .. (360.22,92.5) .. controls (358.65,92.5) and (357.39,89.96) .. (357.39,86.82) .. controls (356.66,78.77) and (357.69,70.49) .. (360.22,64.07) .. controls (363.15,57.25) and (367.25,53.37) .. (371.53,53.37) .. controls (375.81,53.37) and (379.9,57.25) .. (382.84,64.07) .. controls (385.37,70.49) and (386.4,78.77) .. (385.67,86.82) .. controls (385.67,89.96) and (384.4,92.5) .. (382.84,92.5) .. controls (381.28,92.5) and (380.01,89.96) .. (380.01,86.82) .. controls (379.29,78.77) and (380.32,70.49) .. (382.84,64.07) .. controls (385.78,57.25) and (389.87,53.37) .. (394.16,53.37) .. controls (398.44,53.37) and (402.53,57.25) .. (405.47,64.07) .. controls (408,70.49) and (409.03,78.77) .. (408.3,86.82) .. controls (408.3,89.96) and (407.03,92.5) .. (405.47,92.5) .. controls (403.91,92.5) and (402.64,89.96) .. (402.64,86.82) .. controls (401.91,78.77) and (402.94,70.49) .. (405.47,64.07) .. controls (408.75,55.86) and (414.2,52.52) .. (419.2,55.66) .. controls (424.2,58.8) and (427.73,67.78) .. (428.1,78.29) -- (453.55,78.29) ;
%Shape: Capacitor [id:dp06797025402687573] 
\draw   (312.13,78.29) -- (312.13,141.93) (332.13,156.07) -- (292.13,156.07) (332.13,141.93) -- (292.13,141.93) (312.13,156.07) -- (312.13,219.71) ;
%Shape: Circle [id:dp7868953093861399] 
\draw   (524.71,149) .. controls (524.71,135.19) and (513.52,124) .. (499.71,124) .. controls (485.9,124) and (474.71,135.19) .. (474.71,149) .. controls (474.71,162.81) and (485.9,174) .. (499.71,174) .. controls (513.52,174) and (524.71,162.81) .. (524.71,149) -- cycle ;
%Straight Lines [id:da8939369324633372] 
\draw    (499.26,78.29) -- (499.26,124) ;
%Straight Lines [id:da9069568537016492] 
\draw    (499.26,174) -- (499.26,219.71) ;
%Straight Lines [id:da4948249174831607] 
\draw    (453.55,78.29) -- (499.26,78.29) ;
%Straight Lines [id:da05322608838796361] 
\draw    (312.13,219.71) -- (499.26,219.71) ;

% Text Node
\draw (125,142) node [anchor=south] [inner sep=0.75pt]   [align=left] {\begin{minipage}[lt]{8.68pt}\setlength\topsep{0pt}
\begin{center}
+
\end{center}

\end{minipage}};
% Text Node
\draw (125,158) node [anchor=north] [inner sep=0.75pt]   [align=left] {\begin{minipage}[lt]{8.67pt}\setlength\topsep{0pt}
\begin{center}
\mbox{-}
\end{center}

\end{minipage}};
% Text Node
\draw (241.42,95.9) node [anchor=north] [inner sep=0.75pt]    {$.1j$};
% Text Node
\draw (290.13,141.93) node [anchor=east] [inner sep=0.75pt]    {$-\frac{1}{3} j$};
% Text Node
\draw (382.84,95.9) node [anchor=north] [inner sep=0.75pt]    {$.1j$};
% Text Node
\draw (499.71,142) node [anchor=south] [inner sep=0.75pt]  [xscale=-1] [align=left] {\begin{minipage}[lt]{8.68pt}\setlength\topsep{0pt}
\begin{center}
+
\end{center}

\end{minipage}};
% Text Node
\draw (499.71,158) node [anchor=north] [inner sep=0.75pt]  [xscale=-1] [align=left] {\begin{minipage}[lt]{8.67pt}\setlength\topsep{0pt}
\begin{center}
\mbox{-}
\end{center}

\end{minipage}};
% Text Node
\draw (125,149) node    {$1\angle 0^{\circ }$};
% Text Node
\draw (526.71,149) node [anchor=west] [inner sep=0.75pt]    {$\frac{\sqrt{3}}{3} \angle -30^{\circ }$};
% Text Node
\draw (312.13,74.89) node [anchor=south] [inner sep=0.75pt]    {$a'$};


\end{tikzpicture}

        \caption{Single Phase Equivalent Circuit}
        \label{fig:1}
      \end{figure}

      We can write equations using the voltages at $a'$:

      $$\frac{V_{an}-V_{a'n}}{.1j}+\frac{V_{a''n}-V_{a'n}}{.1j}+\frac{3V_{a'n}}{j}=0$$

      We move $V_{a'n}$ to one side:

      $$\frac{17V_{a'n}}{j}=\frac{10V_{an}}{j}+\frac{10V_{a''n}}{j}$$
      $$V_{a'n}=\frac{10(V_{an}+V_{a''n})}{17}$$
      $$V_{a'n}=\frac{10(1+.5-.2887j)}{17}$$
      $$V_{a'n}=\frac{15-2.887j}{17}$$
      $$V_{a'n}=.8824-.1698j$$

      Converting back to phasor form, we find:

      $$\hat{V}_{a'n}=.8986\angle-10.9^{\circ}[\si{\volt}]$$

      Given that the rest of the voltages are offset by $120^{\circ}$, we may write:

      $$\boxed{\left[ \begin{matrix} V_{a'n}\\V_{b'n}\\V_{c'n} \end{matrix}\right]=.8986\angle\left[ \begin{matrix} -10.9\\-130.9\\109.1\end{matrix} \right]^{\circ}}\left[ \si{\volt} \right]$$

      From here, we finally find $V_{a'b'}$ since we know this is the difference between $V_{a'n}$ and $V_{b'n}$:

      $$V_{a'b'}=V_{a'n}-V_{b'n}$$
      $$V_{a'b'}=.8986\left( \angle-10.9^{\circ}-\angle-130.9^{\circ} \right)$$
      $$V_{a'b'}=.8986\left[ (.982-.1891j)-(-.6547-.7559j)\right]$$
      $$V_{a'b'}=.8986\left[ 1.6367+.5668j\right]$$
      $$V_{a'b'}=1.4707+.5093j$$

      And finally we convert to phasor form:

      $$\boxed{\hat{V}_{a'b'}=1.5564\angle19.1^{\circ}[\si{\volt}]}$$

  \item

    We begin by, as always, converting to a `Y' configuration:

    $$Z_C\to \frac{Z_C}{3}=-\frac{10}{3}j[\si{\ohm}]$$

    This leads to a parallel combination, the equivalent impedance of which we may calculate:

    $$Z_{eq}=\frac{-(10/3)(10)}{10-(10/3)}j=-5j[\si{\ohm}]$$

    The circuit can then be phase-simplified by drawing:

    \begin{figure}[H]
      \centering
      \tikzset{every picture/.style={line width=0.75pt}} %set default line width to 0.75pt        

\begin{tikzpicture}[x=0.75pt,y=0.75pt,yscale=-1,xscale=1]
%uncomment if require: \path (0,437); %set diagram left start at 0, and has height of 437

%Shape: Circle [id:dp25415500658791657] 
\draw   (100,149) .. controls (100,135.19) and (111.19,124) .. (125,124) .. controls (138.81,124) and (150,135.19) .. (150,149) .. controls (150,162.81) and (138.81,174) .. (125,174) .. controls (111.19,174) and (100,162.81) .. (100,149) -- cycle ;
%Straight Lines [id:da8534411808673741] 
\draw    (125,78.29) -- (125,124) ;
%Straight Lines [id:da9051744751164938] 
\draw    (125,174) -- (125,219.71) ;
%Straight Lines [id:da03528415731907275] 
\draw    (170.71,78.29) -- (125,78.29) ;
%Shape: Inductor (Air Core) [id:dp5589699497346402] 
\draw   (170.71,78.29) -- (196.17,78.29) .. controls (196.53,67.78) and (200.07,58.8) .. (205.07,55.66) .. controls (210.07,52.52) and (215.52,55.86) .. (218.79,64.07) .. controls (221.32,70.49) and (222.35,78.77) .. (221.62,86.82) .. controls (221.62,89.96) and (220.35,92.5) .. (218.79,92.5) .. controls (217.23,92.5) and (215.97,89.96) .. (215.97,86.82) .. controls (215.24,78.77) and (216.27,70.49) .. (218.79,64.07) .. controls (221.73,57.25) and (225.83,53.37) .. (230.11,53.37) .. controls (234.39,53.37) and (238.48,57.25) .. (241.42,64.07) .. controls (243.95,70.49) and (244.98,78.77) .. (244.25,86.82) .. controls (244.25,89.96) and (242.98,92.5) .. (241.42,92.5) .. controls (239.86,92.5) and (238.59,89.96) .. (238.59,86.82) .. controls (237.86,78.77) and (238.9,70.49) .. (241.42,64.07) .. controls (244.36,57.25) and (248.45,53.37) .. (252.74,53.37) .. controls (257.02,53.37) and (261.11,57.25) .. (264.05,64.07) .. controls (266.57,70.49) and (267.61,78.77) .. (266.88,86.82) .. controls (266.88,89.96) and (265.61,92.5) .. (264.05,92.5) .. controls (262.49,92.5) and (261.22,89.96) .. (261.22,86.82) .. controls (260.49,78.77) and (261.52,70.49) .. (264.05,64.07) .. controls (267.33,55.86) and (272.77,52.52) .. (277.78,55.66) .. controls (282.78,58.8) and (286.31,67.78) .. (286.68,78.29) -- (312.13,78.29) ;
%Straight Lines [id:da5290413238311188] 
\draw    (312.13,219.71) -- (125,219.71) ;
%Shape: Capacitor [id:dp06797025402687573] 
\draw   (312.13,78.29) -- (312.13,141.93) (332.13,156.07) -- (292.13,156.07) (332.13,141.93) -- (292.13,141.93) (312.13,156.07) -- (312.13,219.71) ;

% Text Node
\draw (125,142) node [anchor=south] [inner sep=0.75pt]   [align=left] {\begin{minipage}[lt]{8.68pt}\setlength\topsep{0pt}
\begin{center}
+
\end{center}

\end{minipage}};
% Text Node
\draw (125,158) node [anchor=north] [inner sep=0.75pt]   [align=left] {\begin{minipage}[lt]{8.67pt}\setlength\topsep{0pt}
\begin{center}
\mbox{-}
\end{center}

\end{minipage}};
% Text Node
\draw (241.42,95.9) node [anchor=north] [inner sep=0.75pt]    {$1j$};
% Text Node
\draw (290.13,141.93) node [anchor=east] [inner sep=0.75pt]    {$-5j$};
% Text Node
\draw (125,149) node    {$1\angle 0^{\circ }$};


\end{tikzpicture}

      \caption{Single Phase Equivalent Circuit}
      \label{fig:2}
    \end{figure}

    Summing impedances, the total becomes:

    $$Z_t=-4j[\si{\ohm}]$$

    The line current can then be calculated as:

    $$I_a=\frac{1}{-4j}$$
    $$\hat{I}_a=\frac{(1\angle0^{\circ})j}{4}$$
    $$\boxed{\hat{I}_a=.25\angle90^{\circ}[\si{\ampere}]}$$

    To find the current through a capacitor, we must find the voltage across one of load phase components:

    $$V_{pha}=(-5j)(.25\angle90^{\circ})$$
    $$V_{pha}=(-5j)(.25j)$$
    $$\hat{V}_{pha}=1.25\angle0^{\circ}[\si{\volt}]$$

    Because of the parallel nature of the capacitor-inductor network, this voltage flows into both branches. To find the current across one capacitor, we find:

    $$I_{cap}=\frac{(\sqrt{3})1.25\angle0^{\circ}}{-10j}$$
    $$I_{cap}=\frac{(\sqrt{3})1.25j}{10}$$
    $$I_{cap}=.2165j$$
    $$\boxed{I_{cap}=.2165\angle90^{\circ}[\si{\ampere}]}$$

    The complex power of the load may be calculated by using the product of the voltage and current phasors. This gives us:

    $$S=3|\hat{V}||\hat{I}|$$
    $$S=3\left(1.25\right)\left(  .25\right)$$
    $$\boxed{S=.9375[\si{\volt\ampere}]}$$

\end{enumerate}

\end{document}

