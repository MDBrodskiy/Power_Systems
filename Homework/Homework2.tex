%%%%%%%%%%%%%%%%%%%%%%%%%%%%%%%%%%%%%%%%%%%%%%%%%%%%%%%%%%%%%%%%%%%%%%%%%%%%%%%%%%%%%%%%%%%%%%%%%%%%%%%%%%%%%%%%%%%%%%%%%%%%%%%%%%%%%%%%%%%%%%%%%%%%%%%%%%%%%%%%%%%
% Written By Michael Brodskiy
% Class: Power Systems Analysis
% Professor: A. Abur
%%%%%%%%%%%%%%%%%%%%%%%%%%%%%%%%%%%%%%%%%%%%%%%%%%%%%%%%%%%%%%%%%%%%%%%%%%%%%%%%%%%%%%%%%%%%%%%%%%%%%%%%%%%%%%%%%%%%%%%%%%%%%%%%%%%%%%%%%%%%%%%%%%%%%%%%%%%%%%%%%%%

\documentclass[12pt]{article} 
\usepackage{alphalph}
\usepackage[utf8]{inputenc}
\usepackage[russian,english]{babel}
\usepackage{titling}
\usepackage{amsmath}
\usepackage{graphicx}
\usepackage{enumitem}
\usepackage{amssymb}
\usepackage[super]{nth}
\usepackage{everysel}
\usepackage{ragged2e}
\usepackage{geometry}
\usepackage{multicol}
\usepackage{fancyhdr}
\usepackage{cancel}
\usepackage{siunitx}
\usepackage{physics}
\usepackage{lastpage}
\usepackage{tikz}
\usepackage{mathdots}
\usepackage{yhmath}
\usepackage{cancel}
\usepackage{color}
\usepackage{array}
\usepackage{multirow}
\usepackage{gensymb}
\usepackage{tabularx}
\usepackage{extarrows}
\usepackage{booktabs}
\usetikzlibrary{fadings}
\usetikzlibrary{patterns}
\usetikzlibrary{shadows.blur}
\usetikzlibrary{shapes}

\geometry{top=1.0in,bottom=1.0in,left=1.0in,right=1.0in}
\newcommand{\subtitle}[1]{%
  \posttitle{%
    \par\end{center}
    \begin{center}\large#1\end{center}
    \vskip0.5em}%

}
\usepackage{hyperref}
\hypersetup{
colorlinks=true,
linkcolor=blue,
filecolor=magenta,      
urlcolor=blue,
citecolor=blue,
}

\pagestyle{fancy}

\lfoot[\vspace{-15pt} \hline]{\vspace{-15pt} \hline}
\rfoot[\vspace{-15pt} \hline]{\vspace{-15pt} \hline}
\cfoot[\thepage]{\thepage}
\chead[\textsc{Electromagnetics}]{\textsc{Electromagnetics}}
\lhead[\textsc{EECE2530/1}]{\textsc{EECE2530/1}}
\rhead[\textsc{Page \thepage \hspace{1pt} of \pageref{LastPage}}]{\textsc{Page \thepage \hspace{1pt} of \pageref{LastPage}}}

\usepackage{float}
\usepackage{listings}
\usepackage{xcolor}
\definecolor{codegreen}{rgb}{0,0.6,0}



\title{Homework 2}
\date{\today}
\author{Michael Brodskiy\\ \small Professor: A. Abur}

\begin{document}

\maketitle

\begin{enumerate}

    \begin{center}
      \underline{Chapter 5}:
    \end{center}

    \setcounter{enumi}{14}

  \item We need to establish a per-unit base. We may begin by choosing $S_{b1}=30[\si{\mega\volt\ampere}]$ and $V_{b1}=13.8[\si{\volt}]$. The impedance in zone 1 will remain the same:

    $$X_{s1}^{new}=X_{s1}^{old}\left( \frac{30}{30} \right)\left( \frac{13.8}{13.8} \right)^2$$
    $$X_{s1}^{new}=.1pu$$

    We adapt the other impedances accordingly:

    $$X_{T1}^{new}=.1\left( \frac{30}{20} \right)\left( \frac{132}{138} \right)^2$$
    $$X_{T1}^{new}=.1372pu$$

    $$X_{T2}^{new}=.12\left( \frac{30}{15} \right)\left( \frac{138}{138} \right)^2$$
    $$X_{T2}^{new}=.24pu$$

    $$X_{s2}^{new}=.08\left( \frac{30}{20} \right)\left( \frac{138}{138} \right)^2$$
    $$X_{s2}^{new}=.12pu$$

    $$Z_{L}^{new}=(20+j100)\left( \frac{30}{138^2} \right)$$
    $$Z_{L}^{new}=(.0315+.15753j)pu$$

    The power of the motor can also be found as:

    $$V_{s2}=\frac{13.8}{13.8}$$
    $$V_{s2}=1pu$$
    $$S_{s2}=\frac{20}{30}$$
    $$S_{s2}=.667pu$$
    
    Using the obtained values, we may generate the following diagram:

    \begin{figure}[H]
      \centering
      \tikzset{every picture/.style={line width=0.75pt}} %set default line width to 0.75pt        

\begin{tikzpicture}[x=0.75pt,y=0.75pt,yscale=-1,xscale=1]
%uncomment if require: \path (0,692); %set diagram left start at 0, and has height of 692

%Shape: Inductor (Air Core) [id:dp19708023217522108] 
\draw   (153,198.79) -- (176.04,198.79) .. controls (176.37,188.28) and (179.57,179.3) .. (184.1,176.15) .. controls (188.62,173.01) and (193.55,176.35) .. (196.52,184.57) .. controls (198.81,190.98) and (199.74,199.27) .. (199.08,207.31) .. controls (199.08,210.45) and (197.93,213) .. (196.52,213) .. controls (195.11,213) and (193.96,210.45) .. (193.96,207.31) .. controls (193.3,199.27) and (194.23,190.98) .. (196.52,184.57) .. controls (199.18,177.74) and (202.88,173.87) .. (206.76,173.87) .. controls (210.64,173.87) and (214.34,177.74) .. (217,184.57) .. controls (219.29,190.98) and (220.22,199.27) .. (219.56,207.31) .. controls (219.56,210.45) and (218.41,213) .. (217,213) .. controls (215.59,213) and (214.44,210.45) .. (214.44,207.31) .. controls (213.78,199.27) and (214.71,190.98) .. (217,184.57) .. controls (219.66,177.74) and (223.36,173.87) .. (227.24,173.87) .. controls (231.12,173.87) and (234.82,177.74) .. (237.48,184.57) .. controls (239.77,190.98) and (240.7,199.27) .. (240.04,207.31) .. controls (240.04,210.45) and (238.89,213) .. (237.48,213) .. controls (236.07,213) and (234.92,210.45) .. (234.92,207.31) .. controls (234.26,199.27) and (235.19,190.98) .. (237.48,184.57) .. controls (240.45,176.35) and (245.38,173.01) .. (249.9,176.15) .. controls (254.43,179.3) and (257.63,188.28) .. (257.96,198.79) -- (281,198.79) ;
%Shape: Inductor (Air Core) [id:dp7336016928016604] 
\draw   (281,198.79) -- (304.04,198.79) .. controls (304.37,188.28) and (307.57,179.3) .. (312.1,176.15) .. controls (316.62,173.01) and (321.55,176.35) .. (324.52,184.57) .. controls (326.81,190.98) and (327.74,199.27) .. (327.08,207.31) .. controls (327.08,210.45) and (325.93,213) .. (324.52,213) .. controls (323.11,213) and (321.96,210.45) .. (321.96,207.31) .. controls (321.3,199.27) and (322.23,190.98) .. (324.52,184.57) .. controls (327.18,177.74) and (330.88,173.87) .. (334.76,173.87) .. controls (338.64,173.87) and (342.34,177.74) .. (345,184.57) .. controls (347.29,190.98) and (348.22,199.27) .. (347.56,207.31) .. controls (347.56,210.45) and (346.41,213) .. (345,213) .. controls (343.59,213) and (342.44,210.45) .. (342.44,207.31) .. controls (341.78,199.27) and (342.71,190.98) .. (345,184.57) .. controls (347.66,177.74) and (351.36,173.87) .. (355.24,173.87) .. controls (359.12,173.87) and (362.82,177.74) .. (365.48,184.57) .. controls (367.77,190.98) and (368.7,199.27) .. (368.04,207.31) .. controls (368.04,210.45) and (366.89,213) .. (365.48,213) .. controls (364.07,213) and (362.92,210.45) .. (362.92,207.31) .. controls (362.26,199.27) and (363.19,190.98) .. (365.48,184.57) .. controls (368.45,176.35) and (373.38,173.01) .. (377.9,176.15) .. controls (382.43,179.3) and (385.63,188.28) .. (385.96,198.79) -- (409,198.79) ;
%Shape: Inductor (Air Core) [id:dp9876624718453518] 
\draw   (409,198.79) -- (432.04,198.79) .. controls (432.37,188.28) and (435.57,179.3) .. (440.1,176.15) .. controls (444.62,173.01) and (449.55,176.35) .. (452.52,184.57) .. controls (454.81,190.98) and (455.74,199.27) .. (455.08,207.31) .. controls (455.08,210.45) and (453.93,213) .. (452.52,213) .. controls (451.11,213) and (449.96,210.45) .. (449.96,207.31) .. controls (449.3,199.27) and (450.23,190.98) .. (452.52,184.57) .. controls (455.18,177.74) and (458.88,173.87) .. (462.76,173.87) .. controls (466.64,173.87) and (470.34,177.74) .. (473,184.57) .. controls (475.29,190.98) and (476.22,199.27) .. (475.56,207.31) .. controls (475.56,210.45) and (474.41,213) .. (473,213) .. controls (471.59,213) and (470.44,210.45) .. (470.44,207.31) .. controls (469.78,199.27) and (470.71,190.98) .. (473,184.57) .. controls (475.66,177.74) and (479.36,173.87) .. (483.24,173.87) .. controls (487.12,173.87) and (490.82,177.74) .. (493.48,184.57) .. controls (495.77,190.98) and (496.7,199.27) .. (496.04,207.31) .. controls (496.04,210.45) and (494.89,213) .. (493.48,213) .. controls (492.07,213) and (490.92,210.45) .. (490.92,207.31) .. controls (490.26,199.27) and (491.19,190.98) .. (493.48,184.57) .. controls (496.45,176.35) and (501.38,173.01) .. (505.9,176.15) .. controls (510.43,179.3) and (513.63,188.28) .. (513.96,198.79) -- (537,198.79) ;
%Straight Lines [id:da11227709832501509] 
\draw    (281,185.57) -- (281,212) ;
%Straight Lines [id:da6474407204450181] 
\draw    (409,185.57) -- (409,212) ;
%Straight Lines [id:da07419771504124362] 
\draw    (153,198.79) -- (96.58,198.79) ;
%Shape: Inductor (Air Core) [id:dp31599427871890895] 
\draw   (96.58,198.79) -- (96.58,215.2) .. controls (107.09,215.44) and (116.07,217.72) .. (119.21,220.94) .. controls (122.35,224.17) and (119.01,227.68) .. (110.79,229.8) .. controls (104.38,231.43) and (96.1,232.09) .. (88.05,231.62) .. controls (84.91,231.62) and (82.36,230.81) .. (82.36,229.8) .. controls (82.36,228.79) and (84.91,227.97) .. (88.05,227.97) .. controls (96.1,227.5) and (104.38,228.17) .. (110.79,229.8) .. controls (117.62,231.69) and (121.49,234.33) .. (121.49,237.1) .. controls (121.49,239.86) and (117.62,242.5) .. (110.79,244.39) .. controls (104.38,246.02) and (96.1,246.69) .. (88.05,246.22) .. controls (84.91,246.22) and (82.36,245.4) .. (82.36,244.39) .. controls (82.36,243.39) and (84.91,242.57) .. (88.05,242.57) .. controls (96.1,242.1) and (104.38,242.76) .. (110.79,244.39) .. controls (117.62,246.29) and (121.49,248.93) .. (121.49,251.69) .. controls (121.49,254.45) and (117.62,257.09) .. (110.79,258.99) .. controls (104.38,260.62) and (96.1,261.28) .. (88.05,260.81) .. controls (84.91,260.81) and (82.36,259.99) .. (82.36,258.99) .. controls (82.36,257.98) and (84.91,257.16) .. (88.05,257.16) .. controls (96.1,256.69) and (104.38,257.36) .. (110.79,258.99) .. controls (119.01,261.1) and (122.35,264.61) .. (119.21,267.84) .. controls (116.07,271.07) and (107.09,273.34) .. (96.58,273.58) -- (96.58,290) ;
%Straight Lines [id:da7285839979509394] 
\draw    (105.92,208.13) -- (87.24,189.44) ;
%Straight Lines [id:da33456051706426315] 
\draw    (584.08,208.13) -- (602.76,189.44) ;
%Straight Lines [id:da45723080388406223] 
\draw    (593.42,198.79) -- (537,198.79) ;
%Shape: Circle [id:dp6911246661193869] 
\draw   (71.58,315) .. controls (71.58,301.19) and (82.77,290) .. (96.58,290) .. controls (110.39,290) and (121.58,301.19) .. (121.58,315) .. controls (121.58,328.81) and (110.39,340) .. (96.58,340) .. controls (82.77,340) and (71.58,328.81) .. (71.58,315) -- cycle ;
%Shape: Inductor (Air Core) [id:dp3615289790026023] 
\draw   (593.42,198.79) -- (593.42,215.2) .. controls (603.93,215.44) and (612.91,217.72) .. (616.05,220.94) .. controls (619.2,224.17) and (615.85,227.68) .. (607.64,229.8) .. controls (601.23,231.43) and (592.94,232.09) .. (584.89,231.62) .. controls (581.75,231.62) and (579.21,230.81) .. (579.21,229.8) .. controls (579.21,228.79) and (581.75,227.97) .. (584.89,227.97) .. controls (592.94,227.5) and (601.23,228.17) .. (607.64,229.8) .. controls (614.47,231.69) and (618.34,234.33) .. (618.34,237.1) .. controls (618.34,239.86) and (614.47,242.5) .. (607.64,244.39) .. controls (601.23,246.02) and (592.94,246.69) .. (584.89,246.22) .. controls (581.75,246.22) and (579.21,245.4) .. (579.21,244.39) .. controls (579.21,243.39) and (581.75,242.57) .. (584.89,242.57) .. controls (592.94,242.1) and (601.23,242.76) .. (607.64,244.39) .. controls (614.47,246.29) and (618.34,248.93) .. (618.34,251.69) .. controls (618.34,254.45) and (614.47,257.09) .. (607.64,258.99) .. controls (601.23,260.62) and (592.94,261.28) .. (584.89,260.81) .. controls (581.75,260.81) and (579.21,259.99) .. (579.21,258.99) .. controls (579.21,257.98) and (581.75,257.16) .. (584.89,257.16) .. controls (592.94,256.69) and (601.23,257.36) .. (607.64,258.99) .. controls (615.85,261.1) and (619.2,264.61) .. (616.05,267.84) .. controls (612.91,271.07) and (603.93,273.34) .. (593.42,273.58) -- (593.42,290) ;
%Shape: Circle [id:dp2055540364649311] 
\draw   (568.42,315) .. controls (568.42,301.19) and (579.61,290) .. (593.42,290) .. controls (607.23,290) and (618.42,301.19) .. (618.42,315) .. controls (618.42,328.81) and (607.23,340) .. (593.42,340) .. controls (579.61,340) and (568.42,328.81) .. (568.42,315) -- cycle ;
%Straight Lines [id:da33204908534721156] 
\draw    (96.58,340) -- (96.58,366.42) ;
%Straight Lines [id:da022036913373745026] 
\draw    (593.42,340) -- (593.42,366.42) ;
%Straight Lines [id:da15777524511695817] 
\draw    (96.58,366.42) -- (593.42,366.42) ;
%Shape: Ground [id:dp38162312909493334] 
\draw   (337,374.75) -- (345,391.42) -- (353,374.75) -- (337,374.75) -- cycle (345,366.42) -- (345,374.75) ;
%Shape: Brace [id:dp4196021809692603] 
\draw   (65,200) .. controls (60.33,200) and (58,202.33) .. (58,207) -- (58,272.5) .. controls (58,279.17) and (55.67,282.5) .. (51,282.5) .. controls (55.67,282.5) and (58,285.83) .. (58,292.5)(58,289.5) -- (58,358) .. controls (58,362.67) and (60.33,365) .. (65,365) ;
%Curve Lines [id:da2979948723480368] 
\draw    (76.58,318) .. controls (93.58,293) and (97.58,346) .. (117.58,315) ;
%Shape: Brace [id:dp3538596417417855] 
\draw   (618.5,366) .. controls (623.17,366) and (625.5,363.67) .. (625.5,359) -- (625.5,293.5) .. controls (625.5,286.83) and (627.83,283.5) .. (632.5,283.5) .. controls (627.83,283.5) and (625.5,280.17) .. (625.5,273.5)(625.5,276.5) -- (625.5,208) .. controls (625.5,203.33) and (623.17,201) .. (618.5,201) ;
%Shape: Brace [id:dp05801140083292433] 
\draw   (280,131) .. controls (280.03,126.33) and (277.71,123.99) .. (273.04,123.96) -- (198.04,123.56) .. controls (191.37,123.52) and (188.05,121.17) .. (188.08,116.5) .. controls (188.05,121.17) and (184.71,123.48) .. (178.04,123.45)(181.04,123.46) -- (103.04,123.04) .. controls (98.37,123.01) and (96.03,125.33) .. (96,130) ;
%Shape: Brace [id:dp5236942092385714] 
\draw   (598,132) .. controls (598.03,127.33) and (595.71,124.99) .. (591.04,124.96) -- (516.04,124.56) .. controls (509.37,124.52) and (506.05,122.17) .. (506.08,117.5) .. controls (506.05,122.17) and (502.71,124.48) .. (496.04,124.45)(499.04,124.46) -- (421.04,124.04) .. controls (416.37,124.01) and (414.03,126.33) .. (414,131) ;
%Shape: Brace [id:dp6518359665358423] 
\draw   (414,131) .. controls (414.03,126.33) and (411.72,123.98) .. (407.05,123.95) -- (357.55,123.58) .. controls (350.88,123.53) and (347.57,121.17) .. (347.61,116.5) .. controls (347.57,121.17) and (344.22,123.48) .. (337.55,123.43)(340.55,123.45) -- (288.05,123.05) .. controls (283.38,123.02) and (281.03,125.33) .. (281,130) ;

% Text Node
\draw (48.5,281.71) node [anchor=south] [inner sep=0.75pt]  [rotate=-270] [align=left] {Generator};
% Text Node
\draw (652,282.71) node [anchor=south] [inner sep=0.75pt]  [rotate=-270] [align=left] {Motor};
% Text Node
\draw (593.42,315) node   [align=left] {M};
% Text Node
\draw (123.58,315) node [anchor=west] [inner sep=0.75pt]    {$ \begin{array}{l}
\text{V} =1pu\\
\text{S} =1pu
\end{array}$};
% Text Node
\draw (123.49,240.5) node [anchor=north west][inner sep=0.75pt]    {$X_{s1} =(.1j)pu$};
% Text Node
\draw (206.76,170.47) node [anchor=south] [inner sep=0.75pt]    {$X_{T1} =(.1372j)pu$};
% Text Node
\draw (355.24,170.47) node [anchor=south] [inner sep=0.75pt]    {$ \begin{array}{l}
Z_{L} =( .0315\\
+.15753j) pu
\end{array}$};
% Text Node
\draw (483.24,170.47) node [anchor=south] [inner sep=0.75pt]    {$X_{T2} =(.24j)pu$};
% Text Node
\draw (578.89,257.16) node [anchor=east] [inner sep=0.75pt]    {$X_{s2} =(.12j)pu$};
% Text Node
\draw (566.42,315) node [anchor=east] [inner sep=0.75pt]    {$ \begin{array}{l}
\text{V} =1pu\\
\text{S} =.667pu
\end{array}$};
% Text Node
\draw (190.37,110.7) node [anchor=south] [inner sep=0.75pt]   [align=left] {Zone 1 ($\displaystyle T_{1}$)};
% Text Node
\draw (508.37,111.7) node [anchor=south] [inner sep=0.75pt]   [align=left] {Zone 3 ($\displaystyle T_{2}$)};
% Text Node
\draw (348.37,110.7) node [anchor=south] [inner sep=0.75pt]   [align=left] {Zone 2 (Line)};


\end{tikzpicture}

      \caption{Impedance Diagram for Problem 1}
      \label{fig:1}
    \end{figure}

  \item

    \begin{enumerate}

      \item 

        We need to begin by calculating the base current. This can be done by doing the following:

        $$I_b=\frac{S_b}{V_b\sqrt{3}}$$
        $$I_b=\frac{30}{13.8\sqrt{3}}$$
        $$I_b=1.2551pu$$

        We then need to find the current at the motor. We start with finding the phase angle:

        $$\cos^{-1}(.85)=31.788^{\circ}$$

        Since the power factor is leading, we get:

        $$\phi=-31.788^{\circ}$$

        The power at the motor becomes:

        $$S=\frac{15}{.85}\angle\phi$$
        $$S=17.647\angle-31.788^{\circ}$$
        $$S=15-9.296j$$

        Converting to per-unit, we get:

        $$S=(0.5-.31j)pu$$

        The motor voltage is:

        $$V_m=\frac{13.2}{13.8}$$
        $$V_m=.9565pu$$

        Thus, we can calculate the current as:

        $$I_m=\frac{S^*}{V}$$
        $$I_m=\frac{.5+.31j}{.9565}$$
        $$\boxed{I_m=(.5227+.3241j)pu}$$

        Since the line is fully serial, the current throughout the line is the same, giving us:

        $$\boxed{I_G=(.5227+.3241j)pu}$$
        $$\boxed{I_L=(.5227+.3241j)pu}$$

        The voltage at the generator terminals can be calculated once we find the overall impedance of the line. This gives:

        $$X_t=.0315+.15753j+.1372j+.24j+.1j+.12j$$
        $$\boxed{X_t=(.0315+.7547j)pu}$$

        The voltage at the terminals is then:

        $$V_G=V_m+IX_t$$
        $$V_G=.9565+(.5227+.3241j)(.0315+.7547j)$$
        $$\boxed{V_G=(0.7284+0.4047j)pu}$$

        We can find the sending-end line voltage by taking the above value and subtracting the voltage drop across $X_g$ and $X_{T1}$:

        $$V_L=V_G-I(X_{s1}+X_{T1})$$
        $$V_L=.7284+.4047j-(.5227+.3241j)[(.1j)+(.1372j)]$$
        $$\boxed{V_L=(.8052+.2807j)pu}$$

        The complex power supplied by the generator may be found by multiplying the voltage and current values together:

        $$S_G=I^*V_G$$
        $$S_G=(.5227-.3241j)(.7284+.4047j)$$
        $$\boxed{S_G=(.5119-.02454j)pu}$$

        In summary, we obtained the following in per-unit:

        $$\boxed{\left\{\begin{array}{l l l} I_m &= & .5227+.3241j\\I_G &= & .5227+.3241j\\I_L &= & .5227+.3241j\\V_G &= & .7284+.4047j\\V_L &= & .8052+.2807j\\S_G &= & .5119-.02454j\end{array}}$$

      \item 

        We may use the per-unit values obtained in (a) to easily convert back to real values:

        $$I=(.5227+.3241j)\left( \frac{30}{13.8} \right)$$
        $$\boxed{I=1.1363+0.7046j[\si{\kilo\ampere}]}$$

        $$V_G=(.7284+.4047j)(13.8)$$
        $$\boxed{V_G=10.0519+5.5849j[\si{\kilo\volt}]}$$

        $$V_L=(.8052+.2807j)(13.8)$$
        $$\boxed{V_L=11.1118+3.8737j[\si{\kilo\volt}]}$$

        $$S_G=(.5119-.02454j)(30)$$
        $$\boxed{S_G=15.357-.7362j[\si{\mega\volt\ampere}]}$$

        In summary, we obtained the following:

        $$\boxed{\left\{\begin{array}{l l l} I_m &= & 1.1363+.7046j [\si{\kilo\ampere}]\\I_G &= & 1.1363+.7046j [\si{\kilo\ampere}]\\I_L &= & 1.1363+.7046j [\si{\kilo\ampere}]\\V_G &= & 10.0519+5.5849j[\si{\kilo\volt}]\\V_L &= & 11.1118+3.8737j[\si{\kilo\volt}]\\S_G &= & 15.357-.7362j[\si{\mega\volt\ampere}]\end{array}}$$

    \end{enumerate}

  \item

    The load impedance may be represented as:

    $$Z_L=14.14+14.14j[\si{\ohm}]$$

    In per-unit, we get:

    $$Z_L=\left( \frac{30}{13.8^2} \right)(14.14+14.14j)$$
    $$Z_L=(2.2275+2.2275j)pu$$

    We then calculate the line current:

    $$I=\frac{V_G}{X_{T1}+X_{T2}+Z_L+Z_{L2}}$$
    $$I=\frac{.9565}{.1372j+.24j+.0315+.15753j+2.2275+2.2275j}$$
    $$\boxed{I=(.1697-.2075j)pu}$$

    Using this, we find the load voltage:

    $$V_L=IZ_L$$
    $$V_L=(.1697-.2075j)(2.2275+2.2275j)$$
    $$\boxed{V_L=(.8402-.0842j)pu}$$

    The voltage then becomes:

    $$V_L=(.840199-.084205)(13.8)$$
    $$\boxed{V_L=11.6-1.162j[\si{\kilo\volt}]}$$

    And finally, the current is:

    $$I_L=\frac{V_L}{Z_L\sqrt{3}}$$
    $$I_L=\frac{(11.5947-1.162j)}{(14.14+14.14j)\sqrt{3}}$$
    $$\boxed{I_L=.213-.2604j[\si{\kilo\ampere}]}$$

    In summary, we obtained the following values:

    $$\boxed{\left\{\begin{array}{lllll} I_L&=& (.1697-.2075j)pu&=&.213-.2604j[\si{\kilo\ampere}]\\ V_L&=& (.8402-.0842j)pu&=&11.6-1.162j[\si{\kilo\volt}]\end{array}}$$

\end{enumerate}

\end{document}

