%%%%%%%%%%%%%%%%%%%%%%%%%%%%%%%%%%%%%%%%%%%%%%%%%%%%%%%%%%%%%%%%%%%%%%%%%%%%%%%%%%%%%%%%%%%%%%%%%%%%%%%%%%%%%%%%%%%%%%%%%%%%%%%%%%%%%%%%%%%%%%%%%%%%%%%%%%%%%%%%%%%
% Written By Michael Brodskiy
% Class: Fundamentals of Electronics
% Professor: A. Ali
%%%%%%%%%%%%%%%%%%%%%%%%%%%%%%%%%%%%%%%%%%%%%%%%%%%%%%%%%%%%%%%%%%%%%%%%%%%%%%%%%%%%%%%%%%%%%%%%%%%%%%%%%%%%%%%%%%%%%%%%%%%%%%%%%%%%%%%%%%%%%%%%%%%%%%%%%%%%%%%%%%%

\documentclass[12pt]{article} 
\usepackage{alphalph}
\usepackage[utf8]{inputenc}
\usepackage[russian,english]{babel}
\usepackage{titling}
\usepackage{amsmath}
\usepackage{graphicx}
\usepackage{enumitem}
\usepackage{amssymb}
\usepackage[super]{nth}
\usepackage{everysel}
\usepackage{ragged2e}
\usepackage{geometry}
\usepackage{multicol}
\usepackage{fancyhdr}
\usepackage{cancel}
\usepackage{siunitx}
\usepackage{physics}
\usepackage{lastpage}
\usepackage{tikz}
\usepackage{mathdots}
\usepackage{yhmath}
\usepackage{cancel}
\usepackage{color}
\usepackage{array}
\usepackage{multirow}
\usepackage{gensymb}
\usepackage{tabularx}
\usepackage{extarrows}
\usepackage{booktabs}
\usetikzlibrary{fadings}
\usetikzlibrary{patterns}
\usetikzlibrary{shadows.blur}
\usetikzlibrary{shapes}

\geometry{top=1.0in,bottom=1.0in,left=1.0in,right=1.0in}
\newcommand{\subtitle}[1]{%
  \posttitle{%
    \par\end{center}
    \begin{center}\large#1\end{center}
    \vskip0.5em}%

}
\usepackage{hyperref}
\hypersetup{
colorlinks=true,
linkcolor=blue,
filecolor=magenta,      
urlcolor=blue,
citecolor=blue,
}

\pagestyle{fancy}

\lfoot[\vspace{-15pt} \hline]{\vspace{-15pt} \hline}
\rfoot[\vspace{-15pt} \hline]{\vspace{-15pt} \hline}
\cfoot[\thepage]{\thepage}
\chead[\textsc{Electromagnetics}]{\textsc{Electromagnetics}}
\lhead[\textsc{EECE2530/1}]{\textsc{EECE2530/1}}
\rhead[\textsc{Page \thepage \hspace{1pt} of \pageref{LastPage}}]{\textsc{Page \thepage \hspace{1pt} of \pageref{LastPage}}}

\usepackage{float}
\usepackage{listings}
\usepackage{xcolor}
\definecolor{codegreen}{rgb}{0,0.6,0}



\title{Lecture 7}
\date{\today}
\author{Michael Brodskiy\\ \small Professor: A. Ali}

\begin{document}

\maketitle

\begin{itemize}

  \item Generator Model

    \begin{itemize}

      \item Synchronous Generator

        \begin{itemize}

          \item Consists of a rotor (rotating piece)

          \item And a stator

          \item Rotating part consists of a coil receiving D.C. current, with a magnetic orientation

          \item The rotation results in a flux, which passes through the stator

          \item From this, we can obtain (with $\lambda=N\phi$):

            $$\frac{d\lambda}{dt}=e=\omega k\sin(\omega t)$$
            $$|\hat{E}_f|=4.44 f N\phi_{max}\propto I_{dc},\omega$$
            $$\lambda=k\cos(\omega t)$$

          \item The terminal voltage may be written as:

            $$\hat{V}_t=\hat{E}_f-(r+jX_s)\hat{I}_a$$

          \item We see that terminal voltage may be controlled using $\hat{E}_f$

          \item The net injected current at different buses (substations) may be calculated using:

            $$\hat{I}_i=\frac{(P_{Gi}+jQ_{Gi})^*}{\hat{V}_i^*}$$

          \item For buses without a generator, there is no source injection or load, so there is no net current injection (zero-injection bus)

            \begin{itemize}

              \item Such substations may be used as 'switching' substations

            \end{itemize}

        \end{itemize}

    \end{itemize}

  \item Transmission Lines

    \begin{itemize}

      \item For a transmission line with resistance $r$ and impedance $jx$, we add two capacitors, one on each side of the impedance, with value $jb$

      \item With this model, we may write:

        $$jb=\frac{1}{2}jb_{lc}$$

        \begin{itemize}

          \item Where $b_{lc}$ is the total line charging susceptance of the line

        \end{itemize}

      \item At a bus with a load, we may find the current as:

        $$\hat{I}_2=\frac{(\hat{V}_2-\hat{V}_1)}{r_{12}+jx_{12}}+\hat{V}_2jb_{12}+\frac{(\hat{V}_2-\hat{V}_3)}{r_{23}+jx_{23}}+\hat{V}_2jb_{23}+\frac{(\hat{V}_2-\hat{V}_4)}{r_{24}+jx_{24}}+\hat{V}_2jb_{24}+$$

    \end{itemize}

  \item Bus Admittance Matrix

    \begin{itemize}

      \item We may build a bus admittance matrix ($Y_{bux}$) may be built by inspection:

        $$Y_{bus}(i,i)=\text{ sum of admittances of all branches incident to bus }i$$
        $$Y_{bus}(i,j)=\text{ negative of the admittance of the branch connecting bus }i\text{ and }j$$
        $$Y_{bus}(i,j)=Y_{bus}(j,i)$$

      \item Properties of $Y_{bus}$:

        \begin{itemize}

          \item Symmetric square matrix

          \item Has complex entries

          \item Super sparse (\textit{i}.\textit{e}. majority of the entries are zero)

          \item $Y_{bus}$ is non-singular only if there exists at least one branch connecting a bus to ground

          \item $(Y_{bus})^{-1}=Z_{bus}$ — the bus impedance matrix!

        \end{itemize}

      \item Properties of $Z_{bus}$:

        \begin{itemize}

          \item It is a full matrix

          \item It is a symmetric matrix

          \item Diagonal entries are equal to the Th\'evenin equivalent impedances looking into the network at that bus

            \begin{itemize}

              \item This means Th\'evenin equivalents may be formulated by finding $Z_{bus}$ and finding the corresponding diagonal entry for a bus

            \end{itemize}

        \end{itemize}

    \end{itemize}

\end{itemize}

\end{document}

