%%%%%%%%%%%%%%%%%%%%%%%%%%%%%%%%%%%%%%%%%%%%%%%%%%%%%%%%%%%%%%%%%%%%%%%%%%%%%%%%%%%%%%%%%%%%%%%%%%%%%%%%%%%%%%%%%%%%%%%%%%%%%%%%%%%%%%%%%%%%%%%%%%%%%%%%%%%%%%%%%%%
% Written By Michael Brodskiy
% Class: Power Systems Analysis
% Professor: A. Ali
%%%%%%%%%%%%%%%%%%%%%%%%%%%%%%%%%%%%%%%%%%%%%%%%%%%%%%%%%%%%%%%%%%%%%%%%%%%%%%%%%%%%%%%%%%%%%%%%%%%%%%%%%%%%%%%%%%%%%%%%%%%%%%%%%%%%%%%%%%%%%%%%%%%%%%%%%%%%%%%%%%%

\documentclass[12pt]{article} 
\usepackage{alphalph}
\usepackage[utf8]{inputenc}
\usepackage[russian,english]{babel}
\usepackage{titling}
\usepackage{amsmath}
\usepackage{graphicx}
\usepackage{enumitem}
\usepackage{amssymb}
\usepackage[super]{nth}
\usepackage{everysel}
\usepackage{ragged2e}
\usepackage{geometry}
\usepackage{multicol}
\usepackage{fancyhdr}
\usepackage{cancel}
\usepackage{siunitx}
\usepackage{physics}
\usepackage{tikz}
\usepackage{mathdots}
\usepackage{yhmath}
\usepackage{cancel}
\usepackage{color}
\usepackage{array}
\usepackage{multirow}
\usepackage{gensymb}
\usepackage{tabularx}
\usepackage{extarrows}
\usepackage{booktabs}
\usepackage{lastpage}
\usepackage{float}
\usepackage{listings}
\usetikzlibrary{fadings}
\usetikzlibrary{patterns}
\usetikzlibrary{shadows.blur}
\usetikzlibrary{shapes}

\geometry{top=1.0in,bottom=1.0in,left=1.0in,right=1.0in}
\newcommand{\subtitle}[1]{%
  \posttitle{%
    \par\end{center}
    \begin{center}\large#1\end{center}
    \vskip0.5em}%

}
\usepackage{hyperref}
\hypersetup{
colorlinks=true,
linkcolor=blue,
filecolor=magenta,      
urlcolor=blue,
citecolor=blue,
}


\title{Lecture 10}
\date{\today}
\author{Michael Brodskiy\\ \small Professor: A. Ali}

\begin{document}

\maketitle

\begin{itemize}

  \item Types of Buses

    \begin{itemize}

      \item Slack Bus (always one)

        \begin{itemize}

          \item Known: Voltage magnitude $V$ and the phase angle $\theta$

          \item Unknown: Net powers $P$ and $Q$

        \end{itemize}

      \item $PV$ (Generator) Buses (there are $n_{PV}$ such buses)

        \begin{itemize}

          \item Known: Voltage magnitude $V$ and real power $P$

            \begin{itemize}

              \item Net powers are:

                $$P=P_G-P_D$$
                $$Q=Q_G-Q_D$$

            \end{itemize}

          \item Unknown: Phase angle $\theta$ and reactive power $Q$

        \end{itemize}

      \item $PQ$ (Load) Buses (there are $n_{PQ}$ such buses)

        \begin{itemize}

          \item Known: Real and reactive powers $P$ and $Q$

            \begin{itemize}

              \item Net powers are:

                $$P=-P_D$$
                $$Q=-Q_D$$

            \end{itemize}

          \item Unknown: Voltage magnitude $V$ and phase angle $\theta$

        \end{itemize}

      \item Note: for these buses $P$ and $Q$ refers to the net injected power (\textit{e}.\textit{g}. If a generator supplies $P_G$ and power $P_D$ is dissipated, then $P$ refers to $P_G-P_D$)

    \end{itemize}

  \item Working with Buses

    \begin{itemize}

      \item Given that $\hat{S}_k=\hat{V}_k\hat{I}_k^*$ and $Y_{bus}\cdot V=I$, we may write:

        $$\hat{I}_k=\sum_{m=1}^n Y_{bus}^*(k,m)\cdot\hat{V}_m^*$$

      \item This gives us:

        $$\hat{S}_k=\hat{V}_k\sum_{m=1}^n Y_{bus}^*(k,m)\cdot\hat{V}^*_m$$

    \end{itemize}

  \item Considering a system with $n$ buses (where $n=1+n_{PV}+n_{PQ}$):

    \begin{itemize}

      \item The total number of equations matches the total number of unknowns

      \item The equations are non-linear and therefore requires solutions of nonlinear algebraic equations

      \item Unknown voltages and phases may be written as:

        $$X^T=[\underbrace{\theta_2,\theta_3,\cdots,\theta_n}_{n-1}\Big| \underbrace{V_{n_{PV}+2},\cdots,V_n}_{n_{PQ}}]$$

      \item For a given bus $k$:

        $$P_k=P_{G}-P_D=V_k\sum V_m(G_{bus}\cos(\theta_{bus})+B_{bus}\sin(\theta_{km})-P_k^{sch}=0\quad\text{ (for $PQ$ and $PV$ buses)}$$
          $$Q_k=V_k\sum V_m(G_{bus}\sin(\theta_{bus})-B_{bus}\cos(\theta_{km})-Q_k^{sch}=0\quad\text{ (for $PQ$ buses)}$$

            \begin{itemize}

              \item We see that the solutions to these equations may be represented by a non-linear vector $X$, which solves $X^T$ from above

            \end{itemize}

    \end{itemize}

  \item Solving Non-linear Algebraic Equations

    \begin{itemize}

      \item We will use the Newton-Raphson method

      \item Scalar case ($F(x)$ is a non-linear function, and $x$ is a single variable):

        \begin{itemize}

          \item We may begin by Taylor expanding around some value $x_o$ such that:

            $$F(x)\approx F(x_o)+\dfrac{\partial F}{\partial x}\Big| (x-x_o)+\overbrace{\cdots}^{\text{higher order terms we ignore}}$$

          \item We let $\dfrac{\partial F}{\partial x}=F_x$ and plug in the value to get:

            $$F_x(x-x_o)+F(x_o)\approx 0$$
            $$x\approx x_o-F_x^{-1}F(x_o)$$

          \item A $k$-iteration counter allows us to write:

            $$x^{k+1}\approx x^k-(F_x^k)^{-1}F(x^k)$$

          \item Starting with $x^1=x_o$, continue updating $x^k$ until $|x^{k+1}-x^k|\leq \varepsilon$ (the $\varepsilon$ is known as the termination threshold)

          \item We may define the Jacobian to rewrite the above formula as:

            $$\bar{x}^{k+1}=\bar{x}^{k}-[J(x^k)]^{-1}F(x^k)$$

        \end{itemize}

    \end{itemize}

  \item Applying this to the case of the power flow problem

    \begin{itemize}

      \item Assume $\hat{x}^o$ and iterate until $\underbrace{||F^{k+1}||}_{\text{norm}|F_j(x)|}< \varepsilon$ (typically $10^{-4}[p.u.]$)

      \item Taking the transpose, we may write:

        $$[F(\bar{x})]^T=\Big[ \underbrace{P_2(x)-P_2^{sch}, P_3(x)-P_3^{sch},\cdots,P_n(x)-P_n^{sch}}_{n}$$
        $$\Big| \underbrace{Q_{n_{PQ}+2}(x)-Q_{n_{PQ}+2}^{sch}, \cdots,Q_n(x)-Q_n^{sch} }_{n_{PQ}}\Big]$$

        \begin{itemize}

          \item The Jacobian matrix will be quite sparse

        \end{itemize}

    \end{itemize}

\end{itemize}

\end{document}

