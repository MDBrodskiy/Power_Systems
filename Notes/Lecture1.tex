%%%%%%%%%%%%%%%%%%%%%%%%%%%%%%%%%%%%%%%%%%%%%%%%%%%%%%%%%%%%%%%%%%%%%%%%%%%%%%%%%%%%%%%%%%%%%%%%%%%%%%%%%%%%%%%%%%%%%%%%%%%%%%%%%%%%%%%%%%%%%%%%%%%%%%%%%%%%%%%%%%%
% Written By Michael Brodskiy
% Class: Fundamentals of Electronics
% Professor: A. Ali
%%%%%%%%%%%%%%%%%%%%%%%%%%%%%%%%%%%%%%%%%%%%%%%%%%%%%%%%%%%%%%%%%%%%%%%%%%%%%%%%%%%%%%%%%%%%%%%%%%%%%%%%%%%%%%%%%%%%%%%%%%%%%%%%%%%%%%%%%%%%%%%%%%%%%%%%%%%%%%%%%%%

\documentclass[12pt]{article} 
\usepackage{alphalph}
\usepackage[utf8]{inputenc}
\usepackage[russian,english]{babel}
\usepackage{titling}
\usepackage{amsmath}
\usepackage{graphicx}
\usepackage{enumitem}
\usepackage{amssymb}
\usepackage[super]{nth}
\usepackage{everysel}
\usepackage{ragged2e}
\usepackage{geometry}
\usepackage{multicol}
\usepackage{fancyhdr}
\usepackage{cancel}
\usepackage{siunitx}
\usepackage{physics}
\usepackage{lastpage}
\usepackage{tikz}
\usepackage{mathdots}
\usepackage{yhmath}
\usepackage{cancel}
\usepackage{color}
\usepackage{array}
\usepackage{multirow}
\usepackage{gensymb}
\usepackage{tabularx}
\usepackage{extarrows}
\usepackage{booktabs}
\usetikzlibrary{fadings}
\usetikzlibrary{patterns}
\usetikzlibrary{shadows.blur}
\usetikzlibrary{shapes}

\geometry{top=1.0in,bottom=1.0in,left=1.0in,right=1.0in}
\newcommand{\subtitle}[1]{%
  \posttitle{%
    \par\end{center}
    \begin{center}\large#1\end{center}
    \vskip0.5em}%

}
\usepackage{hyperref}
\hypersetup{
colorlinks=true,
linkcolor=blue,
filecolor=magenta,      
urlcolor=blue,
citecolor=blue,
}

\pagestyle{fancy}

\lfoot[\vspace{-15pt} \hline]{\vspace{-15pt} \hline}
\rfoot[\vspace{-15pt} \hline]{\vspace{-15pt} \hline}
\cfoot[\thepage]{\thepage}
\chead[\textsc{Electromagnetics}]{\textsc{Electromagnetics}}
\lhead[\textsc{EECE2530/1}]{\textsc{EECE2530/1}}
\rhead[\textsc{Page \thepage \hspace{1pt} of \pageref{LastPage}}]{\textsc{Page \thepage \hspace{1pt} of \pageref{LastPage}}}

\usepackage{float}
\usepackage{listings}
\usepackage{xcolor}
\definecolor{codegreen}{rgb}{0,0.6,0}



\title{Lecture 1}
\date{\today}
\author{Michael Brodskiy\\ \small Professor: A. Ali}

\begin{document}

\maketitle

\begin{itemize}

  \item Linear Circuits

    \begin{itemize}

      \item Circuits which may contain resistors, capacitors, inductors, independent voltage sources, and/or independent current sources

      \item Voltage and current equations take the form of $v(t)=V\sin(\omega t)$ and $I(t)=I\cos(\omega t + \phi)$, respectively

      \item $\omega = 2\pi f$, where $f$ is $60[\si{\hertz}]$ in the United States

      \item All voltages and currents across such components must be sinusoids as well

        \begin{itemize}

          \item $\omega$ stays the same, only the amplitudes ($V$ or $I$) and phase shifts ($\phi$) differ from component to component

          \item To make a more compact representation of these values, we define a phasor, represented as:

            $$V_{avg}(t)=V_{peak}\cos(\omega t + \phi)\to\hat{V}=V_{peak}\angle\phi$$

          \item A phase shift with time difference $\tau$ has $\phi=\omega\tau$

            \begin{itemize}

              \item Note, shifted to the left would mean $\phi>0$, while shifted to the right would mean $\phi<0$

            \end{itemize}

          \item Since operations with phasors often involve integration, we define a root-mean square (RMS) value of a phasor, which gives us:

            $$\hat{V}_1=\frac{V}{\sqrt{2}}\angle\phi$$

            \begin{center}
              and
            \end{center}

            $$\hat{V}_2=\frac{V}{\sqrt{2}}\angle-\phi$$

          \item With $\hat{V}_1$ being a left-shifted phasor and $\hat{V}_2$ being a right-shifted phasor

          \item The RMS is defined with:

            $$V_{rms}=\sqrt{\frac{1}{T}\int_0^T(V\sin(\omega t))^2\,dt}=\frac{1}{\sqrt{2}}V$$

          \item We can say $\hat{V}_1$ is leading $\hat{V}_2$ because it reaches its peak earlier (alternatively, it may be said that $\hat{V}_2$ is lagging $\hat{V}_1$)

        \end{itemize}

      \item Phasor Diagram:

        \begin{figure}[H]
          \centering
          \tikzset{every picture/.style={line width=0.75pt}} %set default line width to 0.75pt        

\begin{tikzpicture}[x=0.75pt,y=0.75pt,yscale=-1,xscale=1]
%uncomment if require: \path (0,300); %set diagram left start at 0, and has height of 300

%Straight Lines [id:da3021496071770512] 
\draw    (191.29,147) -- (330.71,147) ;
\draw [shift={(332.71,147)}, rotate = 180] [color={rgb, 255:red, 0; green, 0; blue, 0 }  ][line width=0.75]    (10.93,-3.29) .. controls (6.95,-1.4) and (3.31,-0.3) .. (0,0) .. controls (3.31,0.3) and (6.95,1.4) .. (10.93,3.29)   ;
%Straight Lines [id:da4669608476883733] 
\draw    (191.29,147) -- (289.88,245.59) ;
\draw [shift={(291.29,247)}, rotate = 225] [color={rgb, 255:red, 0; green, 0; blue, 0 }  ][line width=0.75]    (10.93,-3.29) .. controls (6.95,-1.4) and (3.31,-0.3) .. (0,0) .. controls (3.31,0.3) and (6.95,1.4) .. (10.93,3.29)   ;
%Straight Lines [id:da6867769892767389] 
\draw    (191.29,147) -- (289.88,48.41) ;
\draw [shift={(291.29,47)}, rotate = 135] [color={rgb, 255:red, 0; green, 0; blue, 0 }  ][line width=0.75]    (10.93,-3.29) .. controls (6.95,-1.4) and (3.31,-0.3) .. (0,0) .. controls (3.31,0.3) and (6.95,1.4) .. (10.93,3.29)   ;
%Shape: Arc [id:dp9961821747340037] 
\draw  [draw opacity=0] (212.5,125.79) .. controls (212.5,125.79) and (212.5,125.79) .. (212.5,125.79) .. controls (218.36,131.64) and (221.29,139.32) .. (221.29,147) -- (191.29,147) -- cycle ; \draw   (212.5,125.79) .. controls (212.5,125.79) and (212.5,125.79) .. (212.5,125.79) .. controls (218.36,131.64) and (221.29,139.32) .. (221.29,147) ;  
%Shape: Arc [id:dp8224557686233108] 
\draw  [draw opacity=0] (221.29,147) .. controls (221.29,147) and (221.29,147) .. (221.29,147) .. controls (221.29,155.28) and (217.93,162.78) .. (212.5,168.21) -- (191.29,147) -- cycle ; \draw   (221.29,147) .. controls (221.29,147) and (221.29,147) .. (221.29,147) .. controls (221.29,155.28) and (217.93,162.78) .. (212.5,168.21) ;  

% Text Node
\draw (223.29,143.6) node [anchor=south west] [inner sep=0.75pt]    {$\phi $};
% Text Node
\draw (223.29,150.4) node [anchor=north west][inner sep=0.75pt]    {$\phi $};
% Text Node
\draw (293.29,50.4) node [anchor=north west][inner sep=0.75pt]    {$\hat{V}_{1}$};
% Text Node
\draw (293.29,243.6) node [anchor=south west] [inner sep=0.75pt]    {$\hat{V}_{2}$};
% Text Node
\draw (334.71,147) node [anchor=west] [inner sep=0.75pt]    {$\hat{V}$};


\end{tikzpicture}

          \caption{Sample Phasor Diagram}
          \label{fig:1}
        \end{figure}

    \end{itemize}

  \item Power

    \begin{itemize}

      \item $p(t)=v(t)i(t)$ represents the instantaneous power dissipated (or consumed) by a component, where $v(t)$ is the voltage drop across a component, and $i(t)$ is the current flow through a component

      \item Power follows the passive sign convention, which means that it is defined by the orientation of voltage drop and current flow

        \begin{itemize}

            \item Current must enter the box at the positive voltage terminal to represent consumed power

        \end{itemize}

    \end{itemize}

  \item In a given circuit, let $v(t)=V_{max}\cos(\omega t+\theta_v)$ and $i(t)=I_{max}\cos(\omega t+\theta_I)$, the instantaneous power can be represented as:

    $$p(t)=V_{max}I_{max}\cos(\omega t +\theta_v)\cos(\omega t+\theta_I)$$
    $$p(t)=\frac{V_{max}I_{max}}{2}\left[\cos(\theta_v-\theta_I)+\cos(2\omega t+\theta_v+\theta_I)\right]$$

    \begin{itemize}

      \item It can be observed that the average power may be expressed as:

        $$p_{avg}(t)=\frac{1}{2}V_{max}I_{max}\cos(\theta_v-\theta_I)$$

      \item From here, we can notice that:

        $$\hat{V}=\frac{V_{max}}{\sqrt{2}}\angle\phi_v\text{ and }\hat{I}=\frac{I_{max}}{\sqrt{2}}\angle\phi_I$$

      \item Thus, if we multiply the magnitudes of the two by $\cos$ of the angle difference, we get:

        $$|\hat{V}||\hat{I}|\underbrace{\cos(\theta_v-\theta_I)}_{\text{power factor}}=p_{avg}$$

      \item This can be written as:

        $$p_{avg}=\text{Re}\left\{ \hat{V}_{rms}\hat{I}_{rms}^* \right\}=\frac{1}{2}V_{max}I_{max}\angle\theta_v-\theta_I$$

      \item The reactive power, $Q$, is defined as the imaginary part:

        $$Q=\text{Im}\left\{ \hat{V}_{rms}\hat{I}_{rms}^* \right\}=|\hat{V}_{rms}||\hat{I}_{rms}|\sin(\phi)$$

      \item The apparent (essentially overall) power, $S$ may be written as:

        $$\hat{S}=\hat{V}_{rms}\hat{I}_{rms}$$
        $$|\hat{S}|=|\hat{V}_{rms}||\hat{I}_{rms}|$$

      \item The relationship between the three powers may be expressed as shown below:

        \begin{figure}[H]
          \centering
          \tikzset{every picture/.style={line width=0.75pt}} %set default line width to 0.75pt        

\begin{tikzpicture}[x=0.75pt,y=0.75pt,yscale=-1,xscale=1]
%uncomment if require: \path (0,300); %set diagram left start at 0, and has height of 300

%Straight Lines [id:da3021496071770512] 
\draw    (191.29,147) -- (332.71,147) ;
%Straight Lines [id:da6867769892767389] 
\draw    (191.29,147) -- (332.71,5.58) ;
%Shape: Arc [id:dp9961821747340037] 
\draw  [draw opacity=0] (212.5,125.79) .. controls (212.5,125.79) and (212.5,125.79) .. (212.5,125.79) .. controls (218.36,131.64) and (221.29,139.32) .. (221.29,147) -- (191.29,147) -- cycle ; \draw   (212.5,125.79) .. controls (212.5,125.79) and (212.5,125.79) .. (212.5,125.79) .. controls (218.36,131.64) and (221.29,139.32) .. (221.29,147) ;  
%Straight Lines [id:da11058685813817559] 
\draw    (332.71,147) -- (332.71,5.58) ;

% Text Node
\draw (223.29,137.6) node [anchor=south west] [inner sep=0.75pt]    {$\phi $};
% Text Node
\draw (260,72.89) node [anchor=south east] [inner sep=0.75pt]    {$| \hat{S}| $};
% Text Node
\draw (334.71,8.98) node [anchor=north west][inner sep=0.75pt]    {$Q$};
% Text Node
\draw (334.71,150.4) node [anchor=north west][inner sep=0.75pt]    {$P$};


\end{tikzpicture}

          \caption{Power Relationship}
          \label{fig:2}
        \end{figure}

      \item This lets us derive several equations\footnote{Note: from this point, the RMS subscript is dropped, as it is implied that power relationships always deal with RMS}:

        $$|S|^2=P^2+Q^2$$
        $$|\hat{S}|=|\hat{V}||\hat{I}|$$
        $$\hat{S}=|\hat{V}||\hat{I}|\cos(\phi)+j|\hat{V}||\hat{I}|\sin(\phi)$$

        \begin{itemize}

          \item Although all of these are power values, each generally has its own unit (a bit illogically):

            \begin{center}
              \begin{tabular}[H]{|c|c|}
                \hline
                $P$ & Watts (W)\\
                \hline
                $Q$ & Vars\\
                \hline
                $S$ & Volt-Amp\`eres (VA)\\
                \hline
              \end{tabular}
            \end{center}

        \end{itemize}

    \end{itemize}

  \item Given a network with $\hat{V}$ and $\hat{I}$, then:

    \begin{itemize}

      \item If $\hat{V}$ leads $\hat{I}$, then the power factor is lagging (current is behind the voltage), which results in:

        \begin{itemize}

          \item $\phi>0$

          \item $P>0$

          \item $Q>0$

          \item Thus, real and reactive power are consumed

        \end{itemize}

      \item If $\hat{V}$ lags $\hat{I}$, then the power factor is leading (current is ahead of voltage), which results in:

        \begin{itemize}

          \item $\phi<0$

          \item $P>0$

          \item $Q<0$

          \item Thus, real power is consumed, while reactive power is generated (or drawn)

        \end{itemize}

    \end{itemize}

  \item Given a box shown below, we can make some important derivations:

    \begin{figure}[H]
      \centering
      \tikzset{every picture/.style={line width=0.75pt}} %set default line width to 0.75pt        

\begin{tikzpicture}[x=0.75pt,y=0.75pt,yscale=-1,xscale=1]
%uncomment if require: \path (0,300); %set diagram left start at 0, and has height of 300

%Shape: Rectangle [id:dp6052217164752943] 
\draw   (139,64) -- (239,64) -- (239,210) -- (139,210) -- cycle ;
%Shape: Capacitor [id:dp8313198965554979] 
\draw   (189,97) -- (189,133) (209,141) -- (169,141) (209,133) -- (169,133) (189,141) -- (189,177) ;
%Straight Lines [id:da9264456177024911] 
\draw    (189,97) -- (330.42,97) ;
%Straight Lines [id:da07843275164062369] 
\draw    (189,177) -- (330.42,177) ;
%Straight Lines [id:da17167344062070922] 
\draw    (422.71,82) -- (283.29,82) ;
\draw [shift={(281.29,82)}, rotate = 360] [color={rgb, 255:red, 0; green, 0; blue, 0 }  ][line width=0.75]    (10.93,-3.29) .. controls (6.95,-1.4) and (3.31,-0.3) .. (0,0) .. controls (3.31,0.3) and (6.95,1.4) .. (10.93,3.29)   ;

% Text Node
\draw (330.42,100.4) node [anchor=north] [inner sep=0.75pt]    {$+$};
% Text Node
\draw (330.42,173.6) node [anchor=south] [inner sep=0.75pt]    {$-$};
% Text Node
\draw (330.13,131.4) node [anchor=north] [inner sep=0.75pt]    {$\hat{V}$};
% Text Node
\draw (424.71,85.4) node [anchor=north west][inner sep=0.75pt]    {$\hat{I}$};


\end{tikzpicture}

      \caption{Capacitor Box}
      \label{fig:3}
    \end{figure}

    \begin{itemize}

      \item We know:

        $$i(t)=C\frac{dV_c}{dt}$$

      \item Given:

        $$v_c(t)=V\cos(\omega t)\Rightarrow \hat{V}_c=V\angle0^{\circ}$$
        $$i(t)=CV\omega(-\sin(\omega t)=-\omega CV\cos(\omega t-\pi/2)\Rightarrow \hat{I}=\omega CV\angle\pi/2=j\omega cV$$

        \item The complex power can be defined as:

          $$S=V\angle0^{\circ}(-j\omega CV)=V^2\omega C(-j)$$

        \item As we know, capacitors do not consume real power, and, thus, there is only a reactive component. Thus, we say:

          $$Q=-\omega CV^2$$

        \item We may conclude that capacitors are sources (suppliers) of reactive power (inductors would have the opposite reaction — they would only \textit{consume} reactive power)

    \end{itemize}

  \item Summary

    \begin{itemize}

      \item In time domain:

        $$v(t)=R\cdot i(t)$$
        $$v_L(t)=L\frac{di}{dt}$$
        $$i_C(t)=C\frac{dV_c}{dt}$$
        
      \item In phasor domain:

        $$\hat{V}=R\cdot\hat{I}$$
        $$\hat{V}_L=j\omega L\hat{I},\text{ current lags voltage}$$
        $$\hat{I}_C=j\omega L\hat{V},\text{ current leads voltage}$$

      \item These may be used in combination

        \begin{itemize}

          \item For example, given a resistor and inductor in series, we may write:

            $$v(t)=i(t)\cdot R+L\frac{di}{dt}$$
            $$\hat{V}=\hat{I}(R+j\omega L)$$

        \end{itemize}

      \item Instantaneous power is:

        $$p(t)=v(t)i(t)=\frac{1}{2}V_{max}I_{max}\left[ \cos(\theta_v-\theta_I) +\cos(2\omega t+\theta_v+\theta_I)\right]$$

        \begin{itemize}

          \item Note the frequency of instantaneous power is twice that of voltage or current

        \end{itemize}

      $$p_{avg}=\frac{1}{2}V_{max}I_{max}\underbrace{\cos\small(\overbrace{\theta_v-\theta_I}^{\phi}\small)}_{\text{power factor}}$$

    \item The units of all powers are:

      $$P\to \text{Watts (W)}$$
      $$Q\to \text{Vars}$$
      $$S\to \text{Volt-Amp\`eres (VA)}$$

    \end{itemize}

\end{itemize}

\end{document}

