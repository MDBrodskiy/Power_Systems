%%%%%%%%%%%%%%%%%%%%%%%%%%%%%%%%%%%%%%%%%%%%%%%%%%%%%%%%%%%%%%%%%%%%%%%%%%%%%%%%%%%%%%%%%%%%%%%%%%%%%%%%%%%%%%%%%%%%%%%%%%%%%%%%%%%%%%%%%%%%%%%%%%%%%%%%%%%%%%%%%%%
% Written By Michael Brodskiy
% Class: Power Systems Analysis
% Professor: A. Ali
%%%%%%%%%%%%%%%%%%%%%%%%%%%%%%%%%%%%%%%%%%%%%%%%%%%%%%%%%%%%%%%%%%%%%%%%%%%%%%%%%%%%%%%%%%%%%%%%%%%%%%%%%%%%%%%%%%%%%%%%%%%%%%%%%%%%%%%%%%%%%%%%%%%%%%%%%%%%%%%%%%%

\documentclass[12pt]{article} 
\usepackage{alphalph}
\usepackage[utf8]{inputenc}
\usepackage[russian,english]{babel}
\usepackage{titling}
\usepackage{amsmath}
\usepackage{graphicx}
\usepackage{enumitem}
\usepackage{amssymb}
\usepackage[super]{nth}
\usepackage{everysel}
\usepackage{ragged2e}
\usepackage{geometry}
\usepackage{multicol}
\usepackage{fancyhdr}
\usepackage{cancel}
\usepackage{siunitx}
\usepackage{physics}
\usepackage{tikz}
\usepackage{mathdots}
\usepackage{yhmath}
\usepackage{cancel}
\usepackage{color}
\usepackage{array}
\usepackage{multirow}
\usepackage{gensymb}
\usepackage{tabularx}
\usepackage{extarrows}
\usepackage{booktabs}
\usepackage{lastpage}
\usepackage{float}
\usepackage{listings}
\usetikzlibrary{fadings}
\usetikzlibrary{patterns}
\usetikzlibrary{shadows.blur}
\usetikzlibrary{shapes}

\geometry{top=1.0in,bottom=1.0in,left=1.0in,right=1.0in}
\newcommand{\subtitle}[1]{%
  \posttitle{%
    \par\end{center}
    \begin{center}\large#1\end{center}
    \vskip0.5em}%

}
\usepackage{hyperref}
\hypersetup{
colorlinks=true,
linkcolor=blue,
filecolor=magenta,      
urlcolor=blue,
citecolor=blue,
}


\title{Lecture 8 — Exam 1 Recap}
\date{\today}
\author{Michael Brodskiy\\ \small Professor: A. Ali}

\begin{document}

\maketitle

\begin{itemize}

  \item Consumed real power (average power) from a linear circuit may be expressed as:

    $$P=\frac{1}{2}V_{max}I_{max}\cos(\phi)$$
    $$\phi=\theta_v-\theta_i$$

    \begin{itemize}

      \item With functions:

        $$v(t)=V_{max}\cos(\omega t+\theta_v)$$
        $$i(t)=I_{max}\cos(\omega t+\theta_i)$$

    \end{itemize}

  \item The reactive power can then be defined as:

    $$Q=\frac{1}{2}V_{max}I_{max}\sin(\phi)\quad\text{ (in VArs)}$$

  \item The complex power becomes:

    $$S=P+jQ,\, |S|=\frac{1}{2}V_{max}I_{max}$$

  \item RMS values are define as:

    $$V_{rms}=\frac{1}{\sqrt{2}}V_{max}$$
    $$I_{rms}=\frac{1}{\sqrt{2}}I_{max}$$

  \item The frequency can be written as:

    $$\omega=2\pi f$$

  \item For a capacitor:

    $$v_c(t)i(t)=p_c(t)$$
    $$\text{Ave}\left\{ p_c(t) \right\}=0$$

  \item And for an inductor:

    $$v_l(t)i(t)=p_c(t)$$
    $$\text{Ave}\left\{ p_l(t) \right\}=0$$

  \item Frequency of power is double that of voltage/current

  \item For 3-Phase Circuits:

    $$Q_{3\phi}=\sqrt{3}V_{LL}^{rms}I_L\sin(\phi)$$
    $$P_{3\phi}=3P=3|V_{rms}||I_{rms}|(pf)=\sqrt{3}|V_{LL}^{rms}||I_L|(pf)$$
    $$S_{3\phi}=\sqrt{3}\hat{V}_{LL}^{rms}\hat{I}_{rms}^{*}=3\hat{V}_{LN}\hat{I}_{rms}^{*}$$

  \item Per-unit system (actual/base)

    \begin{itemize}

      \item Select $S_{base}$ for the entire system

      \item Then, select $V_{base}$ for one zone

      \item Calculate $V_{base}$ for all other zones, using transformer turn ratios between them and zone one
        
      \item Calculate $z_{base}$, $I_{base}$ for all zones

        $$I_{base(i)}=\frac{S_{base}}{V_{base(i)}}\text{ and }z_{base}=\frac{V^{2}_{base(i)}}{S_{base}}$$

    \end{itemize}

  \item 3-Phase Per-unit system:

    \begin{itemize}

      \item $I_{base(i)}=\dfrac{S_{base}}{\sqrt{3}V_{LLbase(i)}}$

      \item $z_{base(i)}=\dfrac{V_{LLbase(i)}}{S_{base}}=\dfrac{V_{LNbase(i)}}{S_{base}/3}$

      \item Solving in this manner will give solutions in per-unit

    \end{itemize}

  \item Balanced $3\pi$ circuits

    \begin{itemize}

      \item Sources are balanced, their magnitudes equal, with phase angles $\mp120^{\circ}$ apart

      \item Loads, lines, all impedance/phase will be identical

      \item To convert from delta to 'Y' connection, we may write:

        $$\hat{V}_{an}=\frac{\hat{V}_{AB}}{\sqrt{3}}e^{-j30}$$

      \item Or:

        $$\hat{V}_{AB}=\sqrt{3}\hat{V}_{an}e^{j30}$$

      \item We can find voltages of other phases simply by offsetting the angle of one by $120^{\circ}$
        
      \item We can compensate for $Q$ by adding a capacitor in parallel with the load

      \item The power factor may be computed as: $pf=\cos(\phi)$

    \end{itemize}

\end{itemize}

\end{document}

