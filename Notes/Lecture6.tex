%%%%%%%%%%%%%%%%%%%%%%%%%%%%%%%%%%%%%%%%%%%%%%%%%%%%%%%%%%%%%%%%%%%%%%%%%%%%%%%%%%%%%%%%%%%%%%%%%%%%%%%%%%%%%%%%%%%%%%%%%%%%%%%%%%%%%%%%%%%%%%%%%%%%%%%%%%%%%%%%%%%
% Written By Michael Brodskiy
% Class: Fundamentals of Electronics
% Professor: A. Ali
%%%%%%%%%%%%%%%%%%%%%%%%%%%%%%%%%%%%%%%%%%%%%%%%%%%%%%%%%%%%%%%%%%%%%%%%%%%%%%%%%%%%%%%%%%%%%%%%%%%%%%%%%%%%%%%%%%%%%%%%%%%%%%%%%%%%%%%%%%%%%%%%%%%%%%%%%%%%%%%%%%%

\documentclass[12pt]{article} 
\usepackage{alphalph}
\usepackage[utf8]{inputenc}
\usepackage[russian,english]{babel}
\usepackage{titling}
\usepackage{amsmath}
\usepackage{graphicx}
\usepackage{enumitem}
\usepackage{amssymb}
\usepackage[super]{nth}
\usepackage{everysel}
\usepackage{ragged2e}
\usepackage{geometry}
\usepackage{multicol}
\usepackage{fancyhdr}
\usepackage{cancel}
\usepackage{siunitx}
\usepackage{physics}
\usepackage{tikz}
\usepackage{mathdots}
\usepackage{yhmath}
\usepackage{cancel}
\usepackage{color}
\usepackage{array}
\usepackage{multirow}
\usepackage{gensymb}
\usepackage{tabularx}
\usepackage{extarrows}
\usepackage{booktabs}
\usepackage{lastpage}
\usepackage{float}
\usepackage{listings}
\usetikzlibrary{fadings}
\usetikzlibrary{patterns}
\usetikzlibrary{shadows.blur}
\usetikzlibrary{shapes}

\geometry{top=1.0in,bottom=1.0in,left=1.0in,right=1.0in}
\newcommand{\subtitle}[1]{%
  \posttitle{%
    \par\end{center}
    \begin{center}\large#1\end{center}
    \vskip0.5em}%

}
\usepackage{hyperref}
\hypersetup{
colorlinks=true,
linkcolor=blue,
filecolor=magenta,      
urlcolor=blue,
citecolor=blue,
}


\title{Lecture 6}
\date{\today}
\author{Michael Brodskiy\\ \small Professor: A. Ali}

\begin{document}

\maketitle

\begin{itemize}

  \item Considering a network consisting of a generator, connected to a transmission network, with a distribution network:

    \begin{itemize}

      \item The distribution network is generally low (or lower) voltage area

      \item Distribution networks are generally connected by voltage drop-down transformers

      \item Regulators are implemented — essentially transformers with really small turn radii

        $$\alpha\cong 10-100$$

        \begin{itemize}

          \item Where $\alpha$ is the turns ratio

        \end{itemize}

      \item For a regulator, we may write:

        $$V_{a'n}=V_{an}+\Delta V_{an}$$

      \item There will be a slight phase offset, which may be written as:

        $$\hat{V}_2=\hat{a}\hat{V}_1=\hat{V}_1e^{j\phi}$$

        \begin{itemize}

          \item Where the exponential term is called a ``phase shifter''

        \end{itemize}

      \item A phase shifter may be drawn as follows:

        \begin{figure}[H]
          \centering
          \tikzset{every picture/.style={line width=0.75pt}} %set default line width to 0.75pt        

\begin{tikzpicture}[x=0.75pt,y=0.75pt,yscale=-1,xscale=1]
%uncomment if require: \path (0,423); %set diagram left start at 0, and has height of 423



%Shape: Inductor (Air Core) [id:dp6944404951986608] 
\draw   (344.99,152.25) -- (357.73,174.31) .. controls (367.01,169.38) and (376.56,167.95) .. (381.78,170.71) .. controls (387,173.47) and (386.83,179.86) .. (381.36,186.81) .. controls (377.07,192.21) and (370.41,197.24) .. (363.08,200.64) .. controls (360.36,202.2) and (357.52,202.38) .. (356.74,201.03) .. controls (355.96,199.67) and (357.53,197.3) .. (360.25,195.73) .. controls (366.85,191.08) and (374.54,187.83) .. (381.36,186.81) .. controls (388.74,185.95) and (394.14,187.56) .. (396.28,191.27) .. controls (398.43,194.98) and (397.12,200.46) .. (392.68,206.42) .. controls (388.39,211.82) and (381.73,216.85) .. (374.4,220.24) .. controls (371.68,221.81) and (368.84,221.99) .. (368.06,220.64) .. controls (367.28,219.28) and (368.85,216.91) .. (371.57,215.34) .. controls (378.17,210.69) and (385.86,207.44) .. (392.68,206.42) .. controls (400.06,205.55) and (405.46,207.17) .. (407.61,210.88) .. controls (409.75,214.59) and (408.44,220.07) .. (404,226.03) .. controls (399.71,231.42) and (393.05,236.46) .. (385.72,239.85) .. controls (383,241.42) and (380.16,241.6) .. (379.38,240.25) .. controls (378.6,238.89) and (380.17,236.52) .. (382.89,234.95) .. controls (389.49,230.3) and (397.18,227.05) .. (404,226.03) .. controls (412.76,224.76) and (418.38,227.81) .. (418.16,233.72) .. controls (417.94,239.62) and (411.93,247.18) .. (403.01,252.75) -- (415.75,274.81) ;
%Shape: Inductor (Air Core) [id:dp6830847155502714] 
\draw   (415.75,274.81) -- (390.27,274.81) .. controls (389.91,285.31) and (386.37,294.3) .. (381.37,297.44) .. controls (376.36,300.58) and (370.91,297.24) .. (367.63,289.02) .. controls (365.1,282.61) and (364.07,274.32) .. (364.8,266.28) .. controls (364.8,263.14) and (366.07,260.59) .. (367.63,260.59) .. controls (369.19,260.59) and (370.46,263.14) .. (370.46,266.28) .. controls (371.19,274.32) and (370.16,282.61) .. (367.63,289.02) .. controls (364.69,295.85) and (360.6,299.72) .. (356.31,299.72) .. controls (352.03,299.72) and (347.93,295.85) .. (344.99,289.02) .. controls (342.46,282.61) and (341.43,274.32) .. (342.16,266.28) .. controls (342.16,263.14) and (343.43,260.59) .. (344.99,260.59) .. controls (346.55,260.59) and (347.82,263.14) .. (347.82,266.28) .. controls (348.55,274.32) and (347.52,282.61) .. (344.99,289.02) .. controls (342.05,295.85) and (337.95,299.72) .. (333.67,299.72) .. controls (329.38,299.72) and (325.29,295.85) .. (322.35,289.02) .. controls (319.82,282.61) and (318.79,274.32) .. (319.52,266.28) .. controls (319.52,263.14) and (320.79,260.59) .. (322.35,260.59) .. controls (323.91,260.59) and (325.18,263.14) .. (325.18,266.28) .. controls (325.91,274.32) and (324.88,282.61) .. (322.35,289.02) .. controls (319.07,297.24) and (313.62,300.58) .. (308.61,297.44) .. controls (303.61,294.3) and (300.07,285.31) .. (299.71,274.81) -- (274.23,274.81) ;
%Shape: Inductor (Air Core) [id:dp9921305300764028] 
\draw   (274.23,274.81) -- (286.97,252.75) .. controls (278.05,247.18) and (272.04,239.62) .. (271.82,233.72) .. controls (271.6,227.81) and (277.22,224.76) .. (285.98,226.03) .. controls (292.8,227.05) and (300.49,230.3) .. (307.09,234.95) .. controls (309.81,236.52) and (311.38,238.89) .. (310.6,240.25) .. controls (309.82,241.6) and (306.98,241.42) .. (304.26,239.85) .. controls (296.93,236.46) and (290.27,231.42) .. (285.98,226.03) .. controls (281.54,220.07) and (280.23,214.59) .. (282.37,210.88) .. controls (284.52,207.17) and (289.92,205.55) .. (297.3,206.42) .. controls (304.12,207.44) and (311.81,210.69) .. (318.41,215.34) .. controls (321.13,216.91) and (322.7,219.28) .. (321.92,220.64) .. controls (321.14,221.99) and (318.3,221.81) .. (315.58,220.24) .. controls (308.25,216.85) and (301.59,211.82) .. (297.3,206.42) .. controls (292.86,200.46) and (291.55,194.98) .. (293.69,191.27) .. controls (295.84,187.56) and (301.24,185.95) .. (308.62,186.81) .. controls (315.44,187.83) and (323.13,191.08) .. (329.73,195.73) .. controls (332.45,197.3) and (334.02,199.67) .. (333.24,201.03) .. controls (332.46,202.38) and (329.62,202.2) .. (326.9,200.64) .. controls (319.57,197.24) and (312.91,192.21) .. (308.62,186.81) .. controls (303.14,179.86) and (302.98,173.47) .. (308.2,170.71) .. controls (313.42,167.95) and (322.97,169.38) .. (332.25,174.31) -- (344.99,152.25) ;
%Shape: Inductor (Air Core) [id:dp04754007540369287] 
\draw   (203.48,152.25) -- (228.95,152.25) .. controls (229.31,141.74) and (232.85,132.76) .. (237.85,129.62) .. controls (242.86,126.48) and (248.31,129.82) .. (251.59,138.04) .. controls (254.12,144.45) and (255.15,152.73) .. (254.42,160.78) .. controls (254.42,163.92) and (253.15,166.47) .. (251.59,166.47) .. controls (250.03,166.47) and (248.76,163.92) .. (248.76,160.78) .. controls (248.03,152.73) and (249.06,144.45) .. (251.59,138.04) .. controls (254.53,131.21) and (258.63,127.34) .. (262.91,127.34) .. controls (267.2,127.34) and (271.29,131.21) .. (274.23,138.04) .. controls (276.76,144.45) and (277.79,152.73) .. (277.06,160.78) .. controls (277.06,163.92) and (275.79,166.47) .. (274.23,166.47) .. controls (272.67,166.47) and (271.4,163.92) .. (271.4,160.78) .. controls (270.67,152.73) and (271.7,144.45) .. (274.23,138.04) .. controls (277.17,131.21) and (281.27,127.34) .. (285.55,127.34) .. controls (289.84,127.34) and (293.93,131.21) .. (296.87,138.04) .. controls (299.4,144.45) and (300.43,152.73) .. (299.71,160.78) .. controls (299.71,163.92) and (298.44,166.47) .. (296.87,166.47) .. controls (295.31,166.47) and (294.04,163.92) .. (294.04,160.78) .. controls (293.32,152.73) and (294.35,144.45) .. (296.87,138.04) .. controls (300.15,129.82) and (305.61,126.48) .. (310.61,129.62) .. controls (315.62,132.76) and (319.15,141.74) .. (319.52,152.25) -- (344.99,152.25) ;
%Shape: Inductor (Air Core) [id:dp896520677126098] 
\draw   (415.75,274.81) -- (433.76,256.79) .. controls (426.59,249.11) and (422.74,240.25) .. (424.05,234.49) .. controls (425.37,228.73) and (431.59,227.24) .. (439.72,230.73) .. controls (446.04,233.48) and (452.63,238.61) .. (457.8,244.81) .. controls (460.02,247.03) and (460.92,249.73) .. (459.82,250.84) .. controls (458.72,251.94) and (456.02,251.04) .. (453.8,248.82) .. controls (447.59,243.64) and (442.46,237.05) .. (439.72,230.73) .. controls (436.97,223.82) and (437.13,218.19) .. (440.16,215.16) .. controls (443.19,212.13) and (448.82,211.97) .. (455.73,214.72) .. controls (462.05,217.47) and (468.64,222.6) .. (473.81,228.8) .. controls (476.03,231.02) and (476.94,233.72) .. (475.83,234.82) .. controls (474.73,235.93) and (472.03,235.02) .. (469.81,232.81) .. controls (463.6,227.63) and (458.47,221.04) .. (455.73,214.72) .. controls (452.98,207.81) and (453.14,202.18) .. (456.17,199.15) .. controls (459.2,196.12) and (464.83,195.96) .. (471.74,198.71) .. controls (478.06,201.46) and (484.65,206.59) .. (489.82,212.79) .. controls (492.04,215.01) and (492.95,217.71) .. (491.84,218.81) .. controls (490.74,219.92) and (488.04,219.01) .. (485.82,216.8) .. controls (479.61,211.62) and (474.48,205.03) .. (471.74,198.71) .. controls (468.25,190.58) and (469.74,184.36) .. (475.5,183.05) .. controls (481.26,181.73) and (490.11,185.58) .. (497.8,192.75) -- (515.81,174.74) ;
%Shape: Inductor (Air Core) [id:dp566970111554985] 
\draw   (344.99,397.36) -- (332.25,375.3) .. controls (322.97,380.24) and (313.42,381.67) .. (308.2,378.9) .. controls (302.98,376.14) and (303.14,369.75) .. (308.62,362.8) .. controls (312.91,357.41) and (319.57,352.37) .. (326.9,348.98) .. controls (329.62,347.41) and (332.46,347.23) .. (333.24,348.59) .. controls (334.02,349.94) and (332.45,352.31) .. (329.73,353.88) .. controls (323.13,358.53) and (315.44,361.78) .. (308.62,362.8) .. controls (301.24,363.67) and (295.84,362.06) .. (293.69,358.35) .. controls (291.55,354.63) and (292.86,349.15) .. (297.3,343.19) .. controls (301.59,337.8) and (308.25,332.76) .. (315.58,329.37) .. controls (318.3,327.8) and (321.14,327.62) .. (321.92,328.98) .. controls (322.7,330.33) and (321.13,332.7) .. (318.41,334.27) .. controls (311.81,338.92) and (304.12,342.18) .. (297.3,343.19) .. controls (289.92,344.06) and (284.52,342.45) .. (282.37,338.74) .. controls (280.23,335.03) and (281.54,329.54) .. (285.98,323.58) .. controls (290.27,318.19) and (296.93,313.15) .. (304.26,309.76) .. controls (306.98,308.19) and (309.82,308.01) .. (310.6,309.37) .. controls (311.38,310.72) and (309.81,313.09) .. (307.09,314.66) .. controls (300.49,319.32) and (292.8,322.57) .. (285.98,323.58) .. controls (277.22,324.85) and (271.6,321.8) .. (271.82,315.9) .. controls (272.04,309.99) and (278.05,302.44) .. (286.97,296.87) -- (274.23,274.81) ;
%Shape: Circle [id:dp29106600134650207] 
\draw  [fill={rgb, 255:red, 0; green, 0; blue, 0 }  ,fill opacity=1 ] (199.98,143.75) .. controls (199.98,141.82) and (201.54,140.25) .. (203.48,140.25) .. controls (205.41,140.25) and (206.98,141.82) .. (206.98,143.75) .. controls (206.98,145.68) and (205.41,147.25) .. (203.48,147.25) .. controls (201.54,147.25) and (199.98,145.68) .. (199.98,143.75) -- cycle ;
%Shape: Circle [id:dp6415176882204209] 
\draw  [fill={rgb, 255:red, 0; green, 0; blue, 0 }  ,fill opacity=1 ] (333.99,401.36) .. controls (333.99,399.43) and (335.56,397.86) .. (337.49,397.86) .. controls (339.42,397.86) and (340.99,399.43) .. (340.99,401.36) .. controls (340.99,403.29) and (339.42,404.86) .. (337.49,404.86) .. controls (335.56,404.86) and (333.99,403.29) .. (333.99,401.36) -- cycle ;
%Shape: Circle [id:dp9283759961583852] 
\draw  [fill={rgb, 255:red, 0; green, 0; blue, 0 }  ,fill opacity=1 ] (504.81,170.74) .. controls (504.81,168.81) and (506.38,167.24) .. (508.31,167.24) .. controls (510.25,167.24) and (511.81,168.81) .. (511.81,170.74) .. controls (511.81,172.67) and (510.25,174.24) .. (508.31,174.24) .. controls (506.38,174.24) and (504.81,172.67) .. (504.81,170.74) -- cycle ;
%Shape: Circle [id:dp06506298326718662] 
\draw  [fill={rgb, 255:red, 0; green, 0; blue, 0 }  ,fill opacity=1 ] (327.98,159.75) .. controls (327.98,157.82) and (329.54,156.25) .. (331.48,156.25) .. controls (333.41,156.25) and (334.98,157.82) .. (334.98,159.75) .. controls (334.98,161.68) and (333.41,163.25) .. (331.48,163.25) .. controls (329.54,163.25) and (327.98,161.68) .. (327.98,159.75) -- cycle ;
%Shape: Circle [id:dp3214967464003937] 
\draw  [fill={rgb, 255:red, 0; green, 0; blue, 0 }  ,fill opacity=1 ] (281.98,267.75) .. controls (281.98,265.82) and (283.54,264.25) .. (285.48,264.25) .. controls (287.41,264.25) and (288.98,265.82) .. (288.98,267.75) .. controls (288.98,269.68) and (287.41,271.25) .. (285.48,271.25) .. controls (283.54,271.25) and (281.98,269.68) .. (281.98,267.75) -- cycle ;
%Shape: Circle [id:dp21857579265480143] 
\draw  [fill={rgb, 255:red, 0; green, 0; blue, 0 }  ,fill opacity=1 ] (399.98,267.75) .. controls (399.98,265.82) and (401.54,264.25) .. (403.48,264.25) .. controls (405.41,264.25) and (406.98,265.82) .. (406.98,267.75) .. controls (406.98,269.68) and (405.41,271.25) .. (403.48,271.25) .. controls (401.54,271.25) and (399.98,269.68) .. (399.98,267.75) -- cycle ;

% Text Node
\draw (344.99,148.85) node [anchor=south] [inner sep=0.75pt]    {$a$};
% Text Node
\draw (417.75,278.21) node [anchor=north west][inner sep=0.75pt]    {$b$};
% Text Node
\draw (272.23,278.21) node [anchor=north east] [inner sep=0.75pt]    {$c$};
% Text Node
\draw (203.48,155.65) node [anchor=north] [inner sep=0.75pt]    {$a'$};
% Text Node
\draw (346.99,393.96) node [anchor=south west] [inner sep=0.75pt]    {$c'$};
% Text Node
\draw (517.81,178.14) node [anchor=north west][inner sep=0.75pt]    {$b'$};
% Text Node
\draw (306.2,382.3) node [anchor=north east] [inner sep=0.75pt]    {$-$};
% Text Node
\draw (282.75,350.05) node [anchor=north] [inner sep=0.75pt]  [rotate=-60]  {$-\delta V_{c}$};
% Text Node
\draw (266.35,324) node [anchor=south east] [inner sep=0.75pt]    {$+$};
% Text Node
\draw (237.48,123.73) node [anchor=south east] [inner sep=0.75pt]    {$+$};
% Text Node
\draw (318.82,107.34) node [anchor=north east] [inner sep=0.75pt]    {$-$};
% Text Node
\draw (271.57,106.12) node [anchor=north] [inner sep=0.75pt]    {$-\delta V_{a}$};
% Text Node
\draw (460.23,275.96) node [anchor=south east] [inner sep=0.75pt]  [rotate=-321.84]  {$+$};
% Text Node
\draw (514.05,212.81) node [anchor=north east] [inner sep=0.75pt]  [rotate=-321.84]  {$-$};
% Text Node
\draw (476.15,241.05) node [anchor=north] [inner sep=0.75pt]  [rotate=-321.84]  {$-\delta V_{b}$};


\end{tikzpicture}

          \caption{Phase Shifter Schematic}
          \label{fig:1}
        \end{figure}

      \item Dots added to transformer diagrams indicate that certain points are at the same phase, and show orientations of windings

      \item Phase shift is determined by turns ratio $\to$ small turns ratio, small shift, and vice versa (usually at most a few degrees)

      \item Note that the turns ratio acts as a ``valve'' which can switch reactive power flow from one transformer to the next (for regulators)

        \begin{itemize}

          \item Higher turns in the second transformer would result in higher reactive power

        \end{itemize}

      \item The same ``valve'' logic applies to phase shifters; however, instead of affecting the reactive power this affects real power

      \item In summary: Phase shifter used to change phase and modify real power flow, while regulators modify reactive power flow

    \end{itemize}

  \item Nominal Tap Transformer:

    \begin{itemize}

      \item Modeled by:

        $$\left[ \begin{matrix} i_k\\i_m\end{matrix} \right]=\left[ \begin{matrix} y_t & -y_t\\ -y_t & y_t\end{matrix} \right]\left[\begin{matrix} V_k\\V_m \end{matrix}\right]$$

        \begin{itemize}

          \item Where:

            $$y_t=\frac{1}{r+jx_t}$$

        \end{itemize}

    \end{itemize}

  \item Off-Nominal Tap Transformer:

    \begin{itemize}

      \item Modeled by:

        $$\left[ \begin{matrix} I_1\\I_2\end{matrix} \right]=\left[ \begin{matrix} y_t/a^2 & -y_t/a\\ -y_t/a & y_t\end{matrix} \right]\left[\begin{matrix} V_1\\V_2 \end{matrix}\right]$$

      \item Note that, by setting $a\to 1$, we are essentially returning to a nominal tap transformer, as the regulator component is eliminated

      \item Can be thought of as a regulating transformer equivalent circuit in per-unit

    \end{itemize}

\end{itemize}

\end{document}

