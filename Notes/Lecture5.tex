%%%%%%%%%%%%%%%%%%%%%%%%%%%%%%%%%%%%%%%%%%%%%%%%%%%%%%%%%%%%%%%%%%%%%%%%%%%%%%%%%%%%%%%%%%%%%%%%%%%%%%%%%%%%%%%%%%%%%%%%%%%%%%%%%%%%%%%%%%%%%%%%%%%%%%%%%%%%%%%%%%%
% Written By Michael Brodskiy
% Class: Fundamentals of Electronics
% Professor: A. Ali
%%%%%%%%%%%%%%%%%%%%%%%%%%%%%%%%%%%%%%%%%%%%%%%%%%%%%%%%%%%%%%%%%%%%%%%%%%%%%%%%%%%%%%%%%%%%%%%%%%%%%%%%%%%%%%%%%%%%%%%%%%%%%%%%%%%%%%%%%%%%%%%%%%%%%%%%%%%%%%%%%%%

\documentclass[12pt]{article} 
\usepackage{alphalph}
\usepackage[utf8]{inputenc}
\usepackage[russian,english]{babel}
\usepackage{titling}
\usepackage{amsmath}
\usepackage{graphicx}
\usepackage{enumitem}
\usepackage{amssymb}
\usepackage[super]{nth}
\usepackage{everysel}
\usepackage{ragged2e}
\usepackage{geometry}
\usepackage{multicol}
\usepackage{fancyhdr}
\usepackage{cancel}
\usepackage{siunitx}
\usepackage{physics}
\usepackage{lastpage}
\usepackage{tikz}
\usepackage{mathdots}
\usepackage{yhmath}
\usepackage{cancel}
\usepackage{color}
\usepackage{array}
\usepackage{multirow}
\usepackage{gensymb}
\usepackage{tabularx}
\usepackage{extarrows}
\usepackage{booktabs}
\usetikzlibrary{fadings}
\usetikzlibrary{patterns}
\usetikzlibrary{shadows.blur}
\usetikzlibrary{shapes}

\geometry{top=1.0in,bottom=1.0in,left=1.0in,right=1.0in}
\newcommand{\subtitle}[1]{%
  \posttitle{%
    \par\end{center}
    \begin{center}\large#1\end{center}
    \vskip0.5em}%

}
\usepackage{hyperref}
\hypersetup{
colorlinks=true,
linkcolor=blue,
filecolor=magenta,      
urlcolor=blue,
citecolor=blue,
}

\pagestyle{fancy}

\lfoot[\vspace{-15pt} \hline]{\vspace{-15pt} \hline}
\rfoot[\vspace{-15pt} \hline]{\vspace{-15pt} \hline}
\cfoot[\thepage]{\thepage}
\chead[\textsc{Electromagnetics}]{\textsc{Electromagnetics}}
\lhead[\textsc{EECE2530/1}]{\textsc{EECE2530/1}}
\rhead[\textsc{Page \thepage \hspace{1pt} of \pageref{LastPage}}]{\textsc{Page \thepage \hspace{1pt} of \pageref{LastPage}}}

\usepackage{float}
\usepackage{listings}
\usepackage{xcolor}
\definecolor{codegreen}{rgb}{0,0.6,0}



\title{Lecture 5}
\date{\today}
\author{Michael Brodskiy\\ \small Professor: A. Ali}

\begin{document}

\maketitle

\begin{itemize}

  \item 3-Phase Transformer Bank

    \begin{itemize}

      \item Consists of phases $A, B$, and $C$, each connected to a terminal, which outputs phase $a, b,$ and $c$ at a different voltage

      \item Inside the ``black box'' there are transformers corresponding to each phase, but we are more interested in what occurs at the terminals

    \end{itemize}

  \item With two buses injecting current towards each other, we may derive some equations:

    $$\hat{z}=z\angle\theta_z$$
    $$S_{12}=(V_1\angle\theta_1)(\hat{I}_{12}^*)\quad\text{ and }\quad S_{21}=(V_2\angle\theta_2)(\hat{I}_{21}^*)$$

    \begin{itemize}

      \item This gives us:

        $$S_{12}=\frac{V_1^2}{z}e^{j\theta_z}-\frac{V_1V_2}{z}e^{j\theta_z}e^{j\theta_{12}}$$
        $$S_{12}=C_1-Be^{j\theta_{12}}$$
        $$-S_{21}=-\frac{V_2^2}{z}e^{j\theta_z}+\frac{V_1V_2}{z}e^{j\theta_z}e^{-j\theta_{12}}$$
        $$-S_{21}=C_2+Be^{-j\theta_{12}}$$

      \item These equations allow us to form ``power circle'' graphs to more easily demonstrate how power is delivered from one side of a transmission line to the other

      \item Real power travels long distances better than reactive power

    \end{itemize}

\end{itemize}

\end{document}

