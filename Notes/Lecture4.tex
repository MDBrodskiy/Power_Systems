%%%%%%%%%%%%%%%%%%%%%%%%%%%%%%%%%%%%%%%%%%%%%%%%%%%%%%%%%%%%%%%%%%%%%%%%%%%%%%%%%%%%%%%%%%%%%%%%%%%%%%%%%%%%%%%%%%%%%%%%%%%%%%%%%%%%%%%%%%%%%%%%%%%%%%%%%%%%%%%%%%%
% Written By Michael Brodskiy
% Class: Fundamentals of Electronics
% Professor: A. Ali
%%%%%%%%%%%%%%%%%%%%%%%%%%%%%%%%%%%%%%%%%%%%%%%%%%%%%%%%%%%%%%%%%%%%%%%%%%%%%%%%%%%%%%%%%%%%%%%%%%%%%%%%%%%%%%%%%%%%%%%%%%%%%%%%%%%%%%%%%%%%%%%%%%%%%%%%%%%%%%%%%%%

\documentclass[12pt]{article} 
\usepackage{alphalph}
\usepackage[utf8]{inputenc}
\usepackage[russian,english]{babel}
\usepackage{titling}
\usepackage{amsmath}
\usepackage{graphicx}
\usepackage{enumitem}
\usepackage{amssymb}
\usepackage[super]{nth}
\usepackage{everysel}
\usepackage{ragged2e}
\usepackage{geometry}
\usepackage{multicol}
\usepackage{fancyhdr}
\usepackage{cancel}
\usepackage{siunitx}
\usepackage{physics}
\usepackage{tikz}
\usepackage{mathdots}
\usepackage{yhmath}
\usepackage{cancel}
\usepackage{color}
\usepackage{array}
\usepackage{multirow}
\usepackage{gensymb}
\usepackage{tabularx}
\usepackage{extarrows}
\usepackage{booktabs}
\usepackage{lastpage}
\usepackage{float}
\usepackage{listings}
\usetikzlibrary{fadings}
\usetikzlibrary{patterns}
\usetikzlibrary{shadows.blur}
\usetikzlibrary{shapes}

\geometry{top=1.0in,bottom=1.0in,left=1.0in,right=1.0in}
\newcommand{\subtitle}[1]{%
  \posttitle{%
    \par\end{center}
    \begin{center}\large#1\end{center}
    \vskip0.5em}%

}
\usepackage{hyperref}
\hypersetup{
colorlinks=true,
linkcolor=blue,
filecolor=magenta,      
urlcolor=blue,
citecolor=blue,
}


\title{Lecture 4}
\date{\today}
\author{Michael Brodskiy\\ \small Professor: A. Ali}

\begin{document}

\maketitle

\begin{itemize}

  \item Transformers

    \begin{itemize}

      \item A device which has two coils, a primary and secondary, both of which are physically wrapped around a magnetic rectangular donut-like shape

      \item The magnetic material traps the flux generated by pushing a current through the primary coil

      \item Some flux will ``leak'' into the empty central portion of the shape from both coils, called the ``leakage flux''

      \item Given $N_1$ turns on the primary coil and $N_2$ turns on the secondary coil, the total flux becomes:

        $$N_1\phi=\lambda_1$$
        $$N_2\phi=\lambda_2$$

        $$N_1\frac{d\phi}{dt}=V_1$$
        $$N_2\frac{d\phi}{dt}=V_2$$

        \begin{itemize}

          \item With $\phi$ as the flux through one turn, and $\lambda_n$ as the total flux through the coil

        \end{itemize}

      \item This gives us a voltage ratio:

        $$\frac{V_1}{V_2}=\frac{N_1}{N_2}$$

      \item Thus, transformers are used to step the voltage up/down

      \item In the case that there is no leakage flux (an ideal transformer), the powers are equal: $V_1i_1=V_2i_2$

        $$\frac{V_1}{V_2}=\frac{N_1}{N_2}=\frac{i_2}{i_1}$$

      \item Expressing in matrix notation, we may write:

        $$\left[ \begin{matrix} V_1\\i_1 \end{matrix} \right]=\left[ \begin{matrix} n & 0\\ 0 & \frac{1}{n} \end{matrix}\right]\left[ \begin{matrix} V_2\\ i_2\end{matrix} \right]$$

        \begin{itemize}

          \item With $n=\dfrac{N_1}{N_2}$

        \end{itemize}

      \item To support the model, a ``magnetizing inductor'' (which is non-existent) may be introduced to derive the actual load current

        \begin{itemize}

          \item $i_{\phi}$ is the excitation current, which is typically 1-3\% of the load current

        \end{itemize}

    \end{itemize}

  \item The Per-Unit System

    \begin{itemize}

      \item Per-unit value of a quantity (could be a voltage, a current, impedance, power, etc.) is the actual value divided by the base value

      \item Step 1

        \begin{itemize}

          \item Choose the base value for 2 out of many variables

        \end{itemize}

      \item Step 2

        \begin{itemize}

          \item Determine the base values for all remaining variables

        \end{itemize}

      \item For a change of base in a transformer, we may write:

        $$x_{pu2}=x_{pu1}\left( \frac{V_{b1}}{V_{b2}} \right)^{2}\left( \frac{S_{b2}}{S_{b1}} \right)$$

    \end{itemize}

\end{itemize}

\end{document}

